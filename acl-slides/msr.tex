%%%%%%%%%%%%%%%%%%%%%%%%%%%%%%%%%%%%%%%%%
% Beamer Presentation
% LaTeX Template
% Version 1.0 (10/11/12)
%
% This template has been downloaded from:
% http://www.LaTeXTemplates.com
%
% License:
% CC BY-NC-SA 3.0 (http://creativecommons.org/licenses/by-nc-sa/3.0/)
%
%%%%%%%%%%%%%%%%%%%%%%%%%%%%%%%%%%%%%%%%%

%----------------------------------------------------------------------------------------
%   PACKAGES AND THEMES
%----------------------------------------------------------------------------------------

\documentclass{beamer}

\mode<presentation> {

% The Beamer class comes with a number of default slide themes
% which change the colors and layouts of slides. Below this is a list
% of all the themes, uncomment each in turn to see what they look like.

%\usepackage[utf8]{inputenc}
%\usefonttheme{professionalfonts} 
%\usefonttheme{serif}    
%\usepackage{tgbonum}
%\usepackage[T1]{fontenc}

%\usetheme{default}
%\usetheme{AnnArbor}
%\usetheme{Antibes}
%\usetheme{Bergen}
%\usetheme{Berkeley}
%\usetheme{Berlin}
%\usetheme{Boadilla}
%\usetheme{CambridgeUS}
%\usetheme{Copenhagen}
%\usetheme{Darmstadt}
%\usetheme{Dresden}
%\usetheme{Frankfurt}
%\usetheme{Goettingen}
%\usetheme{Hannover}
%\usetheme{Ilmenau}
%\usetheme{JuanLesPins}
%\usetheme{Luebeck}
\usetheme{Madrid}
%\usetheme{Malmoe}
%\usetheme{Marburg}
%\usetheme{Montpellier}
%\usetheme{PaloAlto}
%\usetheme{Pittsburgh}
%\usetheme{Rochester}
%\usetheme{Singapore}
%\usetheme{Szeged}
%\usetheme{Warsaw}

% As well as themes, the Beamer class has a number of color themes
% for any slide theme. Uncomment each of these in turn to see how it
% changes the colors of your current slide theme.

%\usecolortheme{albatross}
%\usecolortheme{beaver}
%\usecolortheme{beetle}
%\usecolortheme{crane}
%\usecolortheme{dolphin}
%\usecolortheme{dove}
%\usecolortheme{fly}
%\usecolortheme{lily}
%\usecolortheme{orchid}
%\usecolortheme{rose}
%\usecolortheme{seagull}
%\usecolortheme{seahorse}
%\usecolortheme{whale}
%\usecolortheme{wolverine}

%\setbeamertemplate{footline} % To remove the footer line in all slides uncomment this line
%\setbeamertemplate{footline}[page number] % To replace the footer line in all slides with a simple slide count uncomment this line

\setbeamertemplate{navigation symbols}{} % To remove the navigation symbols from the bottom of all slides uncomment this line
}
\usepackage{tikz}
\usetikzlibrary{calc}
\usepackage{graphicx} % Allows including images
\usepackage{booktabs} % Allows the use of \toprule, \midrule and \bottomrule in tables
\usepackage{svg}
\newcommand{\light}[1]{\textcolor{gray}{#1}}
%----------------------------------------------------------------------------------------
%   TITLE PAGE
%----------------------------------------------------------------------------------------

\title[Disaster Summarization]{Predicting Salient Updates for Disaster Summarization}
\author[Chris Kedzie]{Chris Kedzie, Columbia University\\
Kathy McKeown, Columbia University\\
Fernando Diaz, Microsoft Research\\}

\institute[Columbia U.] % Your institution as it will appear on the bottom of every slide, may be shorthand to save space
{
%Computer Science Department \\
%Columbia University \\ % Your institution for the title page
\medskip
\textit{kedzie@cs.columbia.edu} % Your email address
}
\date{\today} % Date, can be changed to a custom date

\newcommand{\tikzmark}[1]{\tikz[overlay,remember picture] \node (#1) {};}
\begin{document}

\begin{frame}
\titlepage % Print the title page as the first slide
\end{frame}

%\begin{frame}
%\frametitle{Overview} % Table of contents slide, comment this block out to remove it
%\tableofcontents % Throughout your presentation, if you choose to use \section{} and \subsection{} commands, these will automatically be printed on this slide as an overview of your presentation
%\end{frame}

%----------------------------------------------------------------------------------------
%   PRESENTATION SLIDES
%----------------------------------------------------------------------------------------

\begin{frame}
\frametitle{It's October 28th, 2012...}
\vspace{10pt}
\includegraphics<2>[scale=0.4]{png/news_anim1}
\includegraphics<3>[scale=0.4]{png/news_anim2}
\includegraphics<4>[scale=0.4]{png/news_anim3}
\includegraphics<5>[scale=0.4]{png/news_anim4}
\includegraphics<6>[scale=0.4]{png/news_anim5}
\end{frame}



\begin{frame}
\frametitle{Information During Disaster}
\begin{itemize}
\item<1-> People turn to the web and mobile services
\item[]
\item<2-> Realtime curation of data requires significant manpower! 
\begin{itemize}
\item<3-> summarizing official information/resources
\item<4-> summarizing eye witness information
\item<5-> correcting/updating erroneous information
\item<6-> predicting likely outcomes
\end{itemize}
\end{itemize}
\end{frame}


\begin{frame}
\frametitle{Information During Disaster}
\begin{itemize}
\item \light{People turn to the web and mobile services}
\item[]
\item \light{Realtime curation of data requires significant manpower!}
\begin{itemize}
\item \textbf{summarizing official information/resources}
\item \light{summarizing eye witness information}
\item \light{correcting/updating erroneous information}
\item \light{predicting likely outcomes}
\end{itemize}
\end{itemize}
\end{frame}


\begin{frame}
\frametitle{Update Summarization}
\textbf{Given an event:} \pause ``guatemala earthquake''
\begin{itemize}
    \item{Monitor a stream of news}
    \item{Identify relevant information}
    \item{Update the user}
        
\end{itemize}
    \pause
    \textbf{CAVEAT:} Updates should be timely and novel! 
\end{frame}


\begin{frame}
    \includegraphics{images/anim1.pdf}
\end{frame}
\begin{frame}
    \includegraphics{images/anim2.pdf}
\end{frame}
\begin{frame}
    \includegraphics{images/anim3.pdf}
\end{frame}
\begin{frame}
    \includegraphics{images/anim4.pdf}
\end{frame}
\begin{frame}
    \includegraphics{images/anim5.pdf}
\end{frame}
\begin{frame}
    \includegraphics{images/anim6.pdf}
\end{frame}
\begin{frame}
    \includegraphics{images/anim7.pdf}
\end{frame}
\begin{frame}
    \includegraphics{images/anim8.pdf}
\end{frame}
\begin{frame}
    \includegraphics{images/anim9.pdf}
\end{frame}
\begin{frame}
    \includegraphics{images/anim10.pdf}
\end{frame}
\begin{frame}
    \includegraphics{images/anim11.pdf}
\end{frame}
\begin{frame}
    \includegraphics{images/anim12.pdf}
\end{frame}
\begin{frame}
    \includegraphics{images/anim13.pdf}
\end{frame}
\begin{frame}
    \includegraphics{images/anim14.pdf}
\end{frame}


\begin{frame}
\frametitle{Characteristics of a good summary of disaster}

$\pmb{+}$ \textbf{Salience} \\
Updates should contain the most important information.\\
~\\

$\pmb{-}$ \textbf{Latency} \\
New information should be delivered as quickly as possible.\\
~\\

$\pmb{-}$ \textbf{Redundancy} \\
Updates should focus on new or updated information
that has not yet been presented to the user.\\
~\\

\end{frame}


\begin{frame}
    \frametitle{Online Update Summarization}
    \textbf{Summarization as Search}\\
    \begin{itemize}
        \item learning 2 search
        \item each search state is a multiclass classification problem 
    \end{itemize}
\end{frame}




\begin{frame}
\frametitle{Update Summarization Approach}
~\\
At time $t$:
\begin{enumerate}
\item Predict salience for input sentences. 
\pause
\begin{itemize}
\item Domain-specific features for predicting salience
\end{itemize}
\pause
\item Select exemplar sentences for $t$ that are both representative and
    salient.
\pause
%\begin{itemize}
%\item Incorporate salience predictions in this decision
%\end{itemize}
\item Remove redundant sentences.
%\pause
\end{enumerate}
\end{frame}

\begin{frame}
    \frametitle{How do we measure salience?}

    \pause

    event: 2012 Guatemala Earthquake
    \tikzmark{nuggets}{
    \begin{itemize}
        \pause
        \item epicenter was located in the Pacific Ocean
        \pause
    \item \alert<8>{tsunami warnings 100-200 miles from epicenter}
        \pause
        \item felt in El Salvador and parts of Mexico
        \item[] $\;\;\;\;\;\;\;\;\;\vdots \;\;\;\;\;\;\;\;\;\;\;\;\;\;\vdots\;\;\;\;\;\;\;\;\;\;\;\;\;\;\;\vdots$
        
    \end{itemize}
    }
    \pause
    \tikz[overlay,remember picture]{
        \draw[draw=red,thick,double,fill opacity=0.2] ($(nuggets)+(-5.6,-0.2)$) rectangle ($(nuggets)+(2.8,-2.7)$);
        \node[draw=red,thick,double,fill=white] at ($(nuggets)+(1.7,-2.7)$) {nuggets}; 
    }

\uncover<7->{\vspace{16pt}
\pause
\textit{Nicaragua's disaster management said it had issued a 
\alert<8>{local tsunami alert.}}\\
\uncover<9->{
~\\
Given a sentence $s$ and a set of gold nuggets $N$, we define the salience 
of a sentence as
$$\operatorname{salience}(s) = \max_{n \in N} \operatorname{sim}(s,n).$$
    }
}
\end{frame}

\begin{frame}
    \frametitle{How do we measure salience?}

    Given a sentence $s$ and a set of gold nuggets $N$, we define the salience 
    of a sentence as
    $$\operatorname{salience}(s) = \max_{n \in N} \operatorname{sim}(s,n).$$

    \pause
    ~\\

    \textbf{Sentence Similarity}\\
    We use the sentence similarity model of \textit{Guo and Diab, (2012)}\\
    \begin{itemize}
        \item model is trained on domain related Wikipedia abstracts
        \item reduces sentences to a low $k$-dimensional latent vector ($k=100$
            in our experiments)
        \item similarity of two sentences is the cosine similarity of their 
            latent vectors
    \end{itemize}
\end{frame}

\begin{frame}
    \frametitle{How do we measure salience?}

    Given a sentence $s \in S$ and a set of gold nuggets $N$, we define the 
    salience of a sentence as
    $$\operatorname{salience}(s) = \max_{n \in N} \operatorname{sim}(s,n).$$

    ~\\
    $$\uncover<2->{s_1 }\uncover<6->{,\;\; s_2} 
         \uncover<9->{,\;\; s_3} 
         \uncover<12->{,\;\; \ldots,\;\; s_k}\uncover<2->{ \sim S}$$
    ~\\
    \begin{center}
    \uncover<3->{
    \begin{tabular}{c | c }
        $y$ & $X$\\[2pt]
    \hline
    \uncover<4->{$\operatorname{salience}(s_1)$} & \uncover<5->{\alert<17>{$\phi(s_1)$}}\\[2pt]
    \uncover<7->{$\operatorname{salience}(s_2)$} & \uncover<8->{\alert<17>{$\phi(s_2)$}}\\[2pt] 
    \uncover<10->{$\operatorname{salience}(s_3)$} & \uncover<11->{\alert<17>{$\phi(s_3)$}}\\[2pt]
    \uncover<13->{$\vdots$} & \uncover<14->{$\vdots$}\\[2pt]    
    \uncover<15->{$\operatorname{salience}(s_k)$} & \uncover<16->{\alert<17>{$\phi(s_k)$}}\\[2pt]
    \end{tabular}
    }
    \end{center}
\end{frame}
\begin{frame}
\frametitle{Predicting Salience: Model Features}
\begin{itemize}
\item Basic sentence level features
\begin{itemize}
\item sentence length
\item punctuation count
\item \alert<2-4>{number of capitalized words}
\item sentence position in document
\end{itemize}
\end{itemize}

\pause
\textbf{High Salience}\\ 
\alert<2>{Nicaragua's} disaster management said it had issued a local tsunami alert.\\
\textbf{Medium Salience} \\
\alert<3>{People} streamed out of homes, schools and office buildings as far north as \alert<3>{Mexico City.}\\

\textbf{Low Salience} \\
\alert<4>{Add} to \alert<4>{Digg Add} to del.icio.us \alert<4>{Add} to \alert<4>{Facebook Add} to \alert<4>{Myspace} 

\end{frame}

\begin{frame}
\frametitle{Predicting Salience: Model Features}
\begin{itemize}
\item Basic sentence level features
\item Query features
\begin{itemize}
\item query term matches
\item event type synonym, \alert<2>{hypernym}, and hyponym matches
\end{itemize}
\end{itemize}

\textbf{High Salience}\\ 
Nicaragua's \alert<2>{disaster} management said it had issued a local tsunami alert.\\
\textbf{Medium Salience} \\
People streamed out of homes, schools and office buildings as far north as 
Mexico City.\\

\textbf{Low Salience} \\
Add to Digg Add to del.icio.us Add to Facebook Add to Myspace

\end{frame}


\begin{frame}
\frametitle{Predicting Salience: Model Features}
\begin{itemize}
\item Basic sentence level features
\item Query features
\item Language Models (5-gram interpolated Kneser-Ney model)
\begin{itemize}
\item \alert<2>{generic news corpus (10 years of AP and NY Times articles)}
\item \alert<4>{domain specific corpus (domain related Wikipedia articles)}
\end{itemize}
\end{itemize}

\pause
\textbf{High Salience}\\ 
\alert<2,4>{Nicaragua's disaster management said it had issued a local tsunami alert.}\\
\textbf{Medium Salience} \\
\alert<2>{People streamed out of homes, schools and office buildings as far north as Mexico City.}\\
\textbf{Low Salience} \\
Add to Digg Add to del.icio.us Add to Facebook Add to Myspace 


\end{frame}


\begin{frame}
\frametitle{Predicting Salience: Model Features}
\begin{itemize}
\item Basic sentence level features
\item Query features
\item Language Models
\item Geographic Features
\begin{itemize}
\item input sentences tagged with Named-Entity tagger
\item coordinates are retrieved for each location in a sentence
\item mean distance to event
\end{itemize}
\end{itemize}

\pause
\textbf{High Salience}\\ 
\alert<2>{Nicaragua's} disaster management said it had issued a local tsunami alert.\\
\textbf{Medium Salience} \\
People streamed out of homes, schools and office buildings as far north as \alert<2>{Mexico City.}\\
\textbf{Low Salience} \\
Add to Digg Add to del.icio.us Add to Facebook Add to Myspace 



\end{frame}

\begin{frame}
\frametitle{Predicting Salience: Model Features}
\begin{itemize}
\item Basic sentence level features
\item Query features
\item Language Models (5-gram interpolated Kneser-Ney model)
\item Geographic Features
\item Temporal Features
\begin{itemize}
\item average tf-idf at the current time $t_0$
\item difference of average tf-idf at time $t_0$ and $t_{-1}$
\item $\;\;\;\;\;\;\;\;\;\vdots$
\item difference of average tf-idf at time $t_0$ and $t_{-24}$
\item hours since event start time
\end{itemize}
\end{itemize}


\pause
\textbf{High Salience}\\ 
\alert<2>{Nicaragua's disaster management} said it had issued a local \alert<2>{tsunami} alert.\\
\textbf{Medium Salience} \\
People streamed out of homes, schools and office buildings as far north as Mexico City.\\
\textbf{Low Salience} \\
\alert<3>{Add to Digg Add to del.icio.us Add to Facebook Add to Myspace} 

\end{frame}

\begin{frame}
\frametitle{Affinity Propagation Clustering}
\begin{itemize}
\pause
\item assigns each data point to an exemplar data point \\
     (these assignments
      define a cluster)
\pause
\item exemplars are actual data points that best represent their cluster
\pause
\item exemplar selection is determined by a similarity matrix $S$ and 
      exemplar \textit{preference}
\pause
\begin{itemize}
\pause 
\item similarities $S$ -- element $s(i,j)$ is the semantic similarity of sentence $i$ to sentence $j$
\pause
\item \textit{preference} -- scalar value that represents 
      apriori how suited a data point is to be an exemplar
\pause
\item \alert<+->{preference $=$ salience}
\pause
\item \alert<+->{exemplar $=$ update}
\end{itemize}

\end{itemize}
%\begin{columns}[T]
%\column{0.5\textwidth}
%$$r(i,k) \leftarrow s(i,k) - \max_{k^\prime\ne k} a(i,k^\prime) 
%+ s(i,k^\prime) $$

%\column{0.5\textwidth}
%    Update 2
%\end{columns}

\end{frame}



\begin{frame}
\frametitle{Affinity Propagation Clustering}
    \includegraphics{images/cluster_anim1.pdf}
\end{frame}
\begin{frame}
\frametitle{Affinity Propagation Clustering}
    \includegraphics{images/cluster_anim2.pdf}
\end{frame}
\begin{frame}
\frametitle{Affinity Propagation Clustering}
    \includegraphics{images/cluster_anim3.pdf}
\end{frame}
\begin{frame}
\frametitle{Affinity Propagation Clustering}
    \includegraphics{images/cluster_anim4.pdf}
\end{frame}
\begin{frame}
\frametitle{Affinity Propagation Clustering}
    \includegraphics{images/cluster_anim5.pdf}
\end{frame}
\begin{frame}
\frametitle{Affinity Propagation Clustering}
    \includegraphics{images/cluster_anim6.pdf}
\end{frame}
\begin{frame}
\frametitle{Affinity Propagation Clustering}
    \includegraphics{images/cluster_anim7.pdf}
\end{frame}


\begin{frame}
\frametitle{Trec Temporal Summarization Track}
\begin{itemize}
\pause
\item TREC KBA Stream Corpus
\pause
\begin{itemize}
\item hourly web crawl
\item October 2011 - February 2013
\item 16.1TB!
\end{itemize}
\pause
\item[]
\item Training Data
\begin{itemize}
\item Training Events
\end{itemize}
\end{itemize}
\end{frame}

\begin{frame}
\frametitle{Trec TS Events}
\begin{itemize}
    \item{Natural Disasters}
        \begin{itemize}
            \item{\textbf{2012 Guatemalan Earthquake}}
            \item{Hurricane Sandy\\}
            $\;\;\;\;\;\;\;\;\vdots$
        \end{itemize}
    \item{Man-made Disasters}
        \begin{itemize}
            \item{2012 Pakistan Garment Factory Fires}
            \item{2012 Buenos Aires Rail Disaster\\}
            $\;\;\;\;\;\;\;\;\vdots$
        \end{itemize}
    \item{Terrorism/Violence}
    \item{Social Unrest}
\end{itemize}
\end{frame}
%\begin{frame}
%put pic of doc frequency over time

\begin{frame}
\frametitle{Trec Temporal Summarization Track}
\begin{itemize}
\item TREC KBA Stream Corpus
\begin{itemize}
\item hourly web crawl
\item October 2011 - February 2013
\item 16.1TB!
\end{itemize}
\item[]
\item Training Data
\begin{itemize}
\item Training Events
\pause
\item Gold Nugget Information
\end{itemize}
\end{itemize}
\end{frame}


\begin{frame}
\frametitle{Gold Nugget Information}
\begin{itemize}
\item epicenter was located in the Pacific Ocean 
%\item[]
\item tsunami warnings 100-200 miles from epicenter
%\item[]
\item felt in El Salvador and parts of Mexico
\end{itemize}
\end{frame}

\begin{frame}
\frametitle{Extracting Gold Nuggets from Wikipedia}
%SYSTEM TIME: 2012-11-07-19\\
\vspace{5pt}
\includegraphics[width=325pt,height=200pt]{png/wp_evo_1}
\end{frame}
\begin{frame}
\frametitle{Extracting Gold Nuggets from Wikipedia}
%SYSTEM TIME: 2012-11-07-20\\
\vspace{5pt}
\includegraphics[width=325pt,height=200pt]{png/wp_evo_2}
\end{frame}
\begin{frame}
\frametitle{Extracting Gold Nuggets from Wikipedia}
%SYSTEM TIME: 2012-11-08-05\\
\vspace{5pt}
\includegraphics[width=325pt,height=200pt]{png/wp_evo_3}
\end{frame}
\begin{frame}
\frametitle{Extracting Gold Nuggets from Wikipedia}
%SYSTEM TIME: 2012-11-08-05\\
\vspace{5pt}
\includegraphics[width=325pt,height=200pt]{png/wp_evo_4}
\end{frame}

\begin{frame}
\frametitle{Extracting Gold Nuggets from Wikipedia}
%SYSTEM TIME: 2012-11-08-05\\
\vspace{5pt}
\includegraphics[width=325pt,height=200pt]{png/wp_evo_3}
\end{frame}

\begin{frame}
\frametitle{Extracting Gold Nuggets from Wikipedia}
%SYSTEM TIME: 2012-11-08-05\\
\vspace{5pt}
\includegraphics[width=325pt,height=200pt]{png/wp_evo_5}
\end{frame}

\begin{frame}
\frametitle{Trec Temporal Summarization Track}
\begin{itemize}
\item TREC KBA Stream Corpus
\begin{itemize}
\item hourly web crawl
\item October 2011 - February 2013
\item 16.1TB!
\end{itemize}
\item[]
\item Training Data
\begin{itemize}
\item Training Events
\item Gold Nugget Information
\end{itemize}
\end{itemize}
\end{frame}




\begin{frame}
    \frametitle{Experiments}
    
    \begin{itemize}
        \pause
        \item leave one out evaluation on 21 TREC TS events
        \pause
        \item 3 events held out for tuning similarity threshold parameters
        \pause
        \item trained salience models for each event
        \pause
        \begin{itemize}
            \item 1000 sentences sampled for each event
            \item at test time, salience prediction is the average of the 
                23 other models
        \end{itemize}
    \end{itemize}
    
\end{frame}

\begin{frame}
    \frametitle{Evaluation Metrics}
    \begin{itemize}
        \pause
        \item ROUGE 
        \pause
        \begin{itemize}
            \item model summary generated by concatenating event nuggets
        \end{itemize}
        \pause
    \item Expected Gain and Comprehensiveness
        $$\mathbb{E}[\mathrm{Gain}] = \frac{|S_n|}{|S|}\;\;\;\;\;\;\;\;\;
            \textrm{Comprehensiveness} = \frac{|S_n|}{|N|}$$
            where \begin{itemize}
                    \pause
            \item[] $S$ is the set of system updates
                    \pause
                \item[] $S_n$ is the set of nuggets mapped to updates in $S$
                    \pause
                \item[] $N$ is the set of nuggets for the event
            \end{itemize}
    \end{itemize}
\end{frame}

\begin{frame}
    \frametitle{Baselines}
    \begin{itemize}
    \item AP+Salience -- full model
    \item AP -- affinity propagation clustering with no salience
    \item HAC -- hierarchical agglomerative clustering
    \item RS -- rank by salience
    \end{itemize}
\end{frame}

\begin{frame}
    \frametitle{ROUGE}
\begin{table}[h]
\centering
% centering table
\begin{tabular}{l c c c}
% creating 10 columns
\multicolumn{4}{c}{ROUGE-1}\\
\hline
\hline
% inserting double-line
$\mathrm{System}$ & $\mathrm{Recall}$ & $\mathrm{Prec.}$ & $\mathrm{F}_1$\\
[0.5ex]
\hline
AP+Salience & $\mathbf{0.282}$ & $\mathbf{0.344}$ & $\mathbf{0.306}$\\
AP          & $0.245$ & $0.285$ & $0.263$ \\
RS          & $0.230$ & $0.271$ & $0.247$ \\
HAC         & $0.169$ & $0.230$ & $0.186$ \\
\hline % inserts single-line
\end{tabular}
~\\[1ex]
~\\
\begin{tabular}{l c c c}
% creating 10 columns
\multicolumn{4}{c}{ROUGE-2}\\
\hline
\hline
% inserting double-line
$\mathrm{System}$ & $\mathrm{Recall}$ & $\mathrm{Prec.}$ & $\mathrm{F}_1$\\[0.5ex]
\hline
\textsc{AP+Salience} & $\mathbf{0.045}$ & $\mathbf{0.056}$ & $\mathbf{0.049}$\\
\textsc{AP}          & $0.033$ & $0.038$ & $0.035$ \\
\textsc{RS}          & $0.031$ & $0.037$ & $0.034$ \\
\textsc{HAC}         & $0.017$ & $0.024$ & $0.019$ \\
\hline % inserts single-line
\end{tabular}
\end{table}
\end{frame}


\begin{frame}
\frametitle{ROUGE-1 over time}
\begin{center}
\includegraphics[]{images/rouge-time.eps}
\end{center}
\end{frame}


\begin{frame}
\frametitle{$\mathbb{E}[\mathrm{Gain}]$ and $\mathrm{Comprehensiveness}$}
\begin{center}
\includegraphics[scale=.7]{images/nuggets-metrics2.eps}
\end{center}
\end{frame}

\section{Conclusion}
\begin{frame}
\frametitle{Future Work}

\begin{itemize}
\item Online update summarization
\pause
\item[]
\item Improved redundancy
\end{itemize}
\end{frame}

\begin{frame}
\frametitle{Links}

\begin{itemize}
\item Trec Temporal Summarization Track \\ 
    \begin{center}\url{http://www.trec-ts.org}\end{center}
\item[]
\item ACL paper\\\begin{center} \url{http://www.columbia.edu/~crk2130}\end{center}
\end{itemize}

\end{frame}

\end{document}
