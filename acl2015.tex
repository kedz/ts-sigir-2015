%
% File acl2015.tex
%
% Contact: car@ir.hit.edu.cn, gdzhou@suda.edu.cn
%%
%% Based on the style files for ACL-2014, which were, in turn,
%% Based on the style files for ACL-2013, which were, in turn,
%% Based on the style files for ACL-2012, which were, in turn,
%% based on the style files for ACL-2011, which were, in turn, 
%% based on the style files for ACL-2010, which were, in turn, 
%% based on the style files for ACL-IJCNLP-2009, which were, in turn,
%% based on the style files for EACL-2009 and IJCNLP-2008...

%% Based on the style files for EACL 2006 by 
%%e.agirre@ehu.es or Sergi.Balari@uab.es
%% and that of ACL 08 by Joakim Nivre and Noah Smith

\documentclass[11pt]{article}
\usepackage{acl2015}
\usepackage{times}
\usepackage{url}
\usepackage{latexsym}

\usepackage[]{algorithm2e}
\usepackage{amssymb}
\usepackage{amsmath}
\usepackage[inline]{enumitem}
\usepackage{color}
\DeclareMathOperator{\corpus}{\mathcal{C}}
\DeclareMathOperator{\doc}{\mathnormal{d}}
\DeclareMathOperator{\sent}{\mathnormal{s}}
\DeclareMathOperator{\order}{\pi}
\DeclareMathOperator{\dtime}{\mathnormal{t}}
\DeclareMathOperator{\hour}{\mathnormal{h}}
\DeclareMathOperator{\hours}{\mathcal{H}}
\DeclareMathOperator{\Sim}{\mathbf{K}}
\DeclareMathOperator{\SMat}{\mathbf{X}}
\DeclareMathOperator{\Pref}{\boldsymbol{\pi}}
\DeclareMathOperator{\Exemp}{\mathnormal{Exemplars}}
\DeclareMathOperator{\Updates}{\mathnormal{Updates}}


\newcommand{\query}[0]{q}
\newcommand{\stime}[0]{t_s}
\newcommand{\etime}[0]{t_e}

\newcommand{\fdadd}[1]{\textcolor{red}{#1}}
\newcommand{\fdcomment}[1]{\textbf{\textcolor{red}{[FD: #1]}}}
\newcommand{\ckcomment}[1]{\textbf{\textcolor{blue}{[CK: #1]}}}
\newcommand{\kmcomment}[1]{\textbf{\textcolor{green}{[KM: #1]}}}
%\setlength\titlebox{5cm}

% You can expand the titlebox if you need extra space
% to show all the authors. Please do not make the titlebox
% smaller than 5cm (the original size); we will check this
% in the camera-ready version and ask you to change it back.


\title{Instructions for ACL-2015 Proceedings}

\author{First Author \\
  Affiliation / Address line 1 \\
  Affiliation / Address line 2 \\
  Affiliation / Address line 3 \\
  {\tt email@domain} \\\And
  Second Author \\
  Affiliation / Address line 1 \\
  Affiliation / Address line 2 \\
  Affiliation / Address line 3 \\
  {\tt email@domain} \\}

\date{}

\begin{document}
\maketitle
\begin{abstract}
During crises such as natural disasters or other human tragedies, information
needs of both civilians and responders often require urgent, specialized
treatment.  
Monitoring and summarizing a text stream
during such an event remains a difficult problem. 
We present a system for update summarization which predicts the salience of 
sentences with respect to an event and then uses these
predictions to directly bias a clustering algorithm for sentence selection,
increasing the novelty of the updates. We use novel, disaster-specific features
for salience prediction, including geo-locations and language models
representing the language of disaster.
Our evaluation on a standard set of retrospective events using ROUGE shows 
that salience prediction provides a significant improvement over 
other approaches.

%The update summarization
%task refers to extracting short, sentence-length updates about a seed event
%from a stream of text data.
%Slow or ineffective information access can significantly impact the
%safety and health of individuals. 

%We evaluate our system on a standard set of retrospective events using
%the ROUGE automatic evaluation measures. 
%We
%demonstrate the effect of different feature groups, and the importance of
%salience prediction for update selection.

\end{abstract}

\section{Introduction}

\section{Introduction}
\label{sec:introduction}
During crises, information is critical for first responders and those caught
in the event.  When the event is significant, as in the case of Hurricane
Sandy, the amount of information produced by traditional news outlets,
government agencies, relief organizations, and social media can vastly
overwhelm those trying to monitor the situation. Methods for identifying,
tracking, and summarizing events from text based input have been explored
extensively
%KM - I think we should consider adding other event summarizers. 
(e.g.,
\cite{allan1998topic,Filatova&Hatzivassiloglou.04a,Wang&al.11}). However,
these experiments were not developed to handle streaming data from  the large and heterogeneous
environment of the modern web. Neither were they developed to address the
information needs that arise during  crisis
situations. 
%KM - this last sentence sounds somewhat negative.
%there is still a need for robust and scalable 
%methods for automatic summarization.

In this paper, we present an update summarization system to track disasters
across time. Our system predicts sentence salience in the context of a
large-scale event, such as a disaster,  and integrates these predictions into
a clustering based multi-document summarization system. We train a regression
model to predict sentence salience and use these predictions to bias the
formation of sentence clusters around more salient regions in the input space
using affinity propagation (AP) clustering.  AP uses the salience predictions
as well as pairwise similarities among input sentences to identify
\emph{exemplar} sentences, which we use as our summary output.  Our approach
differs from other methods of summarization that compute salience by pairwise
comparisons alone, ignoring features of importance that are intrinsic to the
sentences themselves.

%KM - Chris - add a short overview on the kinds of things we measure and
%results so people are encouraged to read ahead. Keep it general.

In addition to the tight integration between clustering and salience
prediction, our approach also exploits knowledge about disaster to determine
salience. Thus, salience does not just represent importance within a set of
documents; it also represents both how typical this sentence is of the input event
type (i.e., disaster, hurricane, tornado) and whether it specifies information
about this particular disaster. We use a set of language models, one for each
disaster type, to measure typicality of the sentence for the current event type. We
use a feature that measures distance of mentioned location from the center of
the disaster to represent its likelihood of referring to the input disaster. 




%Previous work on generating event descriptions and/or multi-document
%summarization has relied on clustering algorithms to find representative
%sentences appropriate for an event summary.  These methods impose a metric
%space on the text data that can make it difficult to incorporate external
%sources of information elegantly -- in this paper we argue that centroid
%sentences are not a priori the best candidates for inclusion in an event
%summary.



The remainder of the paper is organized as follows. We begin with a review of related work in the information retrieval and multi-document summarization literature.  In section
~\ref{sec:background} we give an overview of the semantic similarity,
%KM - Chris can you update this so it reflects the contributions. We don't
%mention semantic similarity. We do mention salience and the disaster specific
%features. Perhaps we should mention the use of a new approach for semantic
%similarity? 
Gaussian process, and affinity propagation algorithms that make up the bulk of
our TS system. 
Next we describe the data with which we build our various models.
Then in section~\ref{sec:approach} we give a high level sketch
of our system, and then explain each component in detail. 


\section{Related Work}
\section{Related Work}
\label{sec:relatedwork}
A principal concern in extractive multi-document summarization is the
selection of salient sentences for inclusion in summary output
\cite{nenkova2012survey}.  This has often been approached as a ranking
problem.
%We broadly conceptualize this decision as either an intrinsic or extrinsic
%sentence evaluation process. Intrinsic approaches evaluate sentences
%individually, possibly by predicting the impact on summary quality using
%sentece level features. 
Sentences have been ranked by the average word probability, average tf-idf
score, and the number of topically related words (topic-signatures in the
summarization literature)
\cite{nenkova2005impact,hovy1998automated,lin2000automated}. The first two
statistics are easily computable from the input sentences, while the third
only requires an additional, generic background corpus.  Another ranking
approach, centroid summarization, involves creating an average bag of words
(BOW) vector, the centroid, from the input sentences and ranking sentences by
their similarity to the centroid \cite{radev2004centroid}.  Graph
\cite{erkan2004lexrank} and clustering
\cite{hatzivassiloglou2001simfinder,mckeown1999towards,siddharthan2004syntactic}
based approaches, on the other hand, make use of pair-wise similarity
comparisons amongst input sentences.  In these models, salient sentences are
more central to the input or cluster, respectively.

%identify salient regions of the input space while simultaneously coping with
%redundancy.  Graph-based algorithms have been used to rank sentences
%Clustering algorithms, e.g., are commonly used to exploit redundancy in
%input. Input sentences are clustered and summaries are generated by selecting
%the most representative sentence from each cluster.  Graph-based models have
%also been used for summarization.  E.g., the LexRank algorithm treats
%sentences as nodes in a graph, where edges are constructed by way of cosine
%similarity between sentence nodes; edges are either continuosly weighted by
%similarity or discrete, existing only when the similarity is above a
%threshold.  The PageRank algorithm is used on the graph to find the most
%important sentence nodes. In both clustering and graph-based approaches,
%sentence salience is largely determined by the pairwise relations between
%sentences.

Supervised learning has also been applied to this task. Model features are
usually derived from human generated summaries, and are non-lexical in nature
(e.g., sentence starting position, number of topic-signatures, number of
unique words, word frequencies). Seminal work in this area has employed naive
Bayes and logistic regression classifiers to identify sentences for summary
inclusion \cite{kupiec1995trainable,conroy2001using}. 

%\fdadd{
Several researchers have recognized the importance of summarization during
natural disasters.  Guo \textit{et al.} developed a system for detecting
novel, relevant, and comprehensive sentences immediately after a natural
disaster \cite{qi:temporal-summarization}.  The method uses a model of
sentence relevance and novelty in order to select appropriate updates.
Training data for regression targets is automatically generated from
retrospective Wikipedia data.  The system is evaluated on news documents
related to 197 natural and human disasters from 2009 to 2011 using variants of
Rouge modified to capture novelty, relevance, and comprehensiveness
\cite{lin2004rouge}.  Wang and Li present a clustering-based approach to
efficiency detect important updates during natural disasters
\cite{wang:update-summarization}.  The algorithm works by hierarchically
clustering sentences online, allowing the system to output a more expressive
narrative structure than Guo \textit{et al.}.  The method is evaluated on
official press releases related to Hurricane Wilma  in 2005 using Rouge score
between the system summary and a manually generated target summary.


%\section{Motivation}
%\section{Motivation}

%KM If we are short on space, not sure how much this section is needed as it's
%somewhat repetitive of intro.

%KM - Chris, could you get some good references to work in the crisis
%informatics space.
Crisis informatics\cite{?} is dedicated to finding methods for sharing the
right information in a timely fashion with relief organizations during a major crisis. With the
increasing impact of climate change, the world is seeing an increase in
disasters such as hurricanes, tornadoes, flooding and typhoons, all of which
cause major damage and wreak havoc with food supplies, housing, and health
issues. Health epidemics, such as the ebola crisis in West Africa, also create
a need for timely information about where problems are greatest. Social and
political 
crises, such as the current situation in Syria, create similar needs for
humanitarian assistance.

At times of crisis, however, social media can overwhelm current information
systems with the quantity of information, much of which is irrelevant,
unnecessary detail, or out of date. Many approaches have focused on
human-in-the-loop approaches \cite{?} to enable people on the ground to update
crisis interfaces with information about needs. Others have ..
%KM - Chris - fill this out a bit.





%KM - PRobably should make a little more inclusive, there are graph based
%models that use methods such lexical similarity to build the graph,
%probabilistic methods (e.g., based on word probabilities, regression or
%topical signatures). There are also methods that use topic segmentation or
%modeling and select sentences for each topic.



Multi-document summarization offers potential for enabling automatic updates of
relevant, salient information at regular intervals. It would provide
information even when human volunteers are unable to and would filter out
unnecessary and irrelevant detail. 
Most generic multi-document summarization approaches involve either sentence 
clustering or ranking sentences according to metrics that measure the salience
of their words. 
When the summarization task
is broad or underspecified,
%clustering is most appropriate in that the data
these methods are appropriate as they let the data
``to speak for itself.''
%This is probably most true when summarization is 
%interpreted to mean exploratory data analysis. 
When there are more constraints on the output, as in query-focused
summarization or the crisis update task we present here,
results are more accurate when
%often easier to design
appropriate features for the ranking approach are used (e.g., amount of word
overlap for the query-focused approach).
%either by rule or learning-to-rank.
The primary contribution of this paper
is a framework for combining both of these approaches using affinity
propagation clustering as well as disaster-specific features.


\ckcomment{Not sure how to present this together, we also essentially evaluate
a salience model by feature ablation for the TS task, but this feels very specific compared to the general framework idea.}

\fdcomment{I guess I think we need to two things in this section (and/or introduction).  1. explain why the problem matters.  2. explain why our approach is compelling.  }

\fdcomment{if easy to generate, would be good to have a single figure for an event that includes 1. the query, 2. the distribution of (filtered) document volume over time, and 3. a sample of the gold nuggets.  this provides the reader with a visual explanation of the problem, the solution, and the difficulty}


\section{Method}
\section{Problem Definition}
\label{sec:problemdefinition}
In order to evaluate a temporal summarization system, we adopt the simulation-based approach used in the TREC Temporal Summarization Track.  We provide a brief overview of the problem.  Details on the formulation can be found in the track overview paper \cite{aslam2013trec}.  

A temporal summarization system takes as input 
\begin{enumerate*}[label=\itshape\alph*\upshape)]
  \item a short query $\query$ defining the event to be tracked (e.g. `Hurricane Sandy'), 
  \item an event category $\eventcategory$ defining the type of event to be tracked (e.g. `hurricane'), 
  \item a stream of time-stamped documents, $(\doc_0, \doc_1,\ldots,\doc_T)$, presented in temporal order, and
        \item an evaluation time period of interest, $(\stime,\etime)$.  
\end{enumerate*}
While processing documents throughout the time period of interest, the system
must output sentences, known as updates, which are \emph{relevant} to the
query, \emph{comprehensive} with respect to the event, \emph{novel} to the
user, and \emph{timely} with respect to when the update occurred.  Precisely
how to measure these properties will be discussed in Section \ref{sec:methods}.
In our version of this problem, we assume that that the system receives one
batch of new sentence-segmented documents every hour throughout the period of
interest.

We address  disaster types such as terrorism and mass shootings (e.g.,
the 2012 shooting in Aurora, Col.), natural disasters (e.g., Hurricane Sandy),
accidents (e.g., the 2012 Pakistan garment factory fire), astronomical
disasters (e.g,. the Chelyabinsk comet in Russia), and social activism (e.g.,
the Arab spring).
\kmcomment{This was as best as I could do on this. These are the example
  categories from the wikipedia page. However you refer in the language model
  description to the event type earthquake which is more specific. Also, in the
  TREC data you refer to event types that are specified in the metadata. So I'm
  wondering how many different categories you have and where they come from.}

\fdcomment{include description of nuggets here.}
\kmcomment{We should jointly discuss what goes in data and what goes in problem
  definition. I have moved event types here because I think knowing how many
  types and examples of them would be helpful upfront. I think where nuggets
  and wikipedia pages go is questionable.}





% include your own bib file like this:
%\bibliographystyle{acl} \bibliography{acl2015}

\begin{thebibliography}{}

\bibitem[\protect\citename{Aho and Ullman}1972]{Aho:72} Alfred~V. Aho and
    Jeffrey~D. Ullman.  \newblock 1972.  \newblock {\em The Theory of Parsing,
    Translation and Compiling}, volume~1.  \newblock Prentice-{Hall}, Englewood
    Cliffs, NJ.

\bibitem[\protect\citename{{American Psychological Association}}1983]{APA:83}
    {American Psychological Association}.  \newblock 1983.  \newblock {\em
    Publications Manual}.  \newblock American Psychological Association,
    Washington, DC.

\bibitem[\protect\citename{{Association for Computing Machinery}}1983]{ACM:83}
    {Association for Computing Machinery}.  \newblock 1983.  \newblock {\em
    Computing Reviews}, 24(11):503--512.

\bibitem[\protect\citename{Chandra \bgroup et al.\egroup }1981]{Chandra:81}
    Ashok~K. Chandra, Dexter~C. Kozen, and Larry~J. Stockmeyer.  \newblock
    1981.  \newblock Alternation.  \newblock {\em Journal of the Association
    for Computing Machinery}, 28(1):114--133.

\bibitem[\protect\citename{Gusfield}1997]{Gusfield:97} Dan Gusfield.  \newblock
    1997.  \newblock {\em Algorithms on Strings, Trees and Sequences}.
    \newblock Cambridge University Press, Cambridge, UK.

\end{thebibliography}

\end{document}
