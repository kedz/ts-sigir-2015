\label{sec:relatedwork}

A principal concern in extractive multi-document summarization is the
selection of salient sentences for inclusion in summary output
\cite{nenkova2012survey}. 
%We conceptualize 
Existing approaches generally fall into  %falling into
one of three categories, each with specific tradeoffs with respect to update 
summarization. 

First, centrality-focused approaches (including graph,
cluster, and centroid methods) are very natural for retrospective analysis in
the sense that they let the data "speak for itself." 
These methods equate salience with centrality, either to the input or some other
aggregate object (i.e. a cluster center or input centroid).
However, they rely
chiefly on
redundancy. When applied to an unfolding event, there may not exist enough
redundant content at the event onset for these methods to exploit.
Once the event onset has passed, however, the redundancy reduction of these 
methods is quite beneficial.

The second category, predictive approaches,
includes ranking and classification based methods.
Sentences have been ranked by the average word probability, average tf-idf
score, and the number of topically related words (topic-signatures in the
summarization literature)
\cite{nenkova2005impact,hovy1998automated,lin2000automated}. The first two
statistics are easily computable from the input sentences, while the third
only requires an additional, generic background corpus.  
In classification based methods, model features are
usually derived from human generated summaries, and are non-lexical in nature
(e.g., sentence starting position, number of topic-signatures, number of
unique words). Seminal work in this area has employed naive
Bayes and logistic regression classifiers to identify sentences for summary
inclusion \cite{kupiec1995trainable,conroy2001using}. 
While these methods are less dependent on redundancy, the expressiveness of
their features is limited. Our model expands on these basic features to 
account for geographic, temporal, and language model features.

The last category includes probabilistic, information theoritic, and set cover
approaches. While these methods are generally focused on producing diverse
summaries, they are difficult to adapt to the streaming setting, where 
we do not necessarily have a fixed summary length and the corpus to be
summarized contains many irrelevent sentences, i.e. there are large
portions of the corpora that we specificly want to avoid. 




% This has often been approached as a ranking
%problem.
%We broadly conceptualize this decision as either an intrinsic or extrinsic
%sentence evaluation process. Intrinsic approaches evaluate sentences
%individually, possibly by predicting the impact on summary quality using
%sentece level features. 
%Sentences have been ranked by the average word probability, average tf-idf
%score, and the number of topically related words (topic-signatures in the
%summarization literature)
%\cite{nenkova2005impact,hovy1998automated,lin2000automated}. The first two
%statistics are easily computable from the input sentences, while the third
%only requires an additional, generic background corpus.  Another ranking
%approach, centroid summarization, involves creating an average bag of words
%(BOW) vector, the centroid, from the input sentences and ranking sentences by
%their similarity to the centroid \cite{radev2004centroid}.  Graph
%\cite{erkan2004lexrank} and clustering
%\cite{hatzivassiloglou2001simfinder,mckeown1999towards,siddharthan2004syntactic}
%based approaches, on the other hand, make use of pair-wise similarity
%comparisons amongst input sentences.  In these models, salient sentences are
%more central to the input or cluster, respectively.

%identify salient regions of the input space while simultaneously coping with
%redundancy.  Graph-based algorithms have been used to rank sentences
%Clustering algorithms, e.g., are commonly used to exploit redundancy in
%input. Input sentences are clustered and summaries are generated by selecting
%the most representative sentence from each cluster.  Graph-based models have
%also been used for summarization.  E.g., the LexRank algorithm treats
%sentences as nodes in a graph, where edges are constructed by way of cosine
%similarity between sentence nodes; edges are either continuosly weighted by
%similarity or discrete, existing only when the similarity is above a
%threshold.  The PageRank algorithm is used on the graph to find the most
%important sentence nodes. In both clustering and graph-based approaches,
%sentence salience is largely determined by the pairwise relations between
%sentences.

%\fdadd{
Several researchers have recognized the importance of summarization during
natural disasters.  Guo \textit{et al.} developed a system for detecting
novel, relevant, and comprehensive sentences immediately after a natural
disaster \cite{qi:temporal-summarization}.  The method uses a model of
sentence relevance and novelty in order to select appropriate updates.
Training data for regression targets is automatically generated from
retrospective Wikipedia data.  The system is evaluated on news documents
related to 197 natural and human disasters from 2009 to 2011 using variants of
Rouge modified to capture novelty, relevance, and comprehensiveness
\cite{lin2004rouge}.  Wang and Li present a clustering-based approach to
efficiency detect important updates during natural disasters
\cite{wang:update-summarization}.  The algorithm works by hierarchically
clustering sentences online, allowing the system to output a more expressive
narrative structure than Guo \textit{et al.}.  The method is evaluated on
official press releases related to Hurricane Wilma  in 2005 using Rouge score
between the system summary and a manually generated target summary. In addition
to natural disasters, we evaluate our system on a variety of event types 
 such as terrorism and mass shootings (e.g., the 2012
shooting in Aurora, Col.), accidents
(e.g., the 2012 Pakistan garment factory fire), and social activism (e.g., the
Arab spring). The size and scope of the events varies considerably. Hurricane
Sandy, for example, affected multiple countries over weeks while the
2012 Aurora shooting was contained to a single location spanning several hours.



%\fdcomment{Glasgow temporal summarization system \cite{mccreadie:temporal-summarization}.}
