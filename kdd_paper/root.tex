% This is "sig-alternate.tex" V2.0 May 2012 This file should be compiled with
% V2.5 of "sig-alternate.cls" May 2012
%
% This example file demonstrates the use of the 'sig-alternate.cls' V2.5
% LaTeX2e document class file. It is for those submitting articles to ACM
% Conference Proceedings WHO DO NOT WISH TO STRICTLY ADHERE TO THE SIGS
% (PUBS-BOARD-ENDORSED) STYLE.  The 'sig-alternate.cls' file will produce a
% similar-looking, albeit, 'tighter' paper resulting in, invariably, fewer
% pages.
%
% ----------------------------------------------------------------------------------------------------------------
% This .tex file (and associated .cls V2.5) produces: 1) The Permission
% Statement 2) The Conference (location) Info information 3) The Copyright
% Line with ACM data 4) NO page numbers
%
% as against the acm_proc_article-sp.cls file which DOES NOT produce 1) thru'
% 3) above.
%
% Using 'sig-alternate.cls' you have control, however, from within the source
% .tex file, over both the CopyrightYear (defaulted to 200X) and the ACM
% Copyright Data (defaulted to X-XXXXX-XX-X/XX/XX).  e.g.
% \CopyrightYear{2007} will cause 2007 to appear in the copyright line.
% \crdata{0-12345-67-8/90/12} will cause 0-12345-67-8/90/12 to appear in the
% copyright line.
%
% ---------------------------------------------------------------------------------------------------------------
% This .tex source is an example which *does* use the .bib file (from which
% the .bbl file % is produced).  REMEMBER HOWEVER: After having produced the
% .bbl file, and prior to final submission, you *NEED* to 'insert' your .bbl
% file into your source .tex file so as to provide ONE 'self-contained' source
% file.
%
% ================= IF YOU HAVE QUESTIONS ======================= Questions
% regarding the SIGS styles, SIGS policies and procedures, Conferences etc.
% should be sent to Adrienne Griscti (griscti@acm.org)
%
% Technical questions _only_ to Gerald Murray (murray@hq.acm.org)
% ===============================================================
%
% For tracking purposes - this is V2.0 - May 2012

\documentclass{sig-alternate} \usepackage{url} \usepackage{color}
\newcommand{\fdadd}[1]{\textcolor{red}{#1}}
\newcommand{\fdcomment}[1]{\textbf{\textcolor{red}{[FD: #1]}}}

\begin{document}

%
% --- Author Metadata here ---
\conferenceinfo{WOODSTOCK}{'97 El Paso, Texas USA}
%\CopyrightYear{2007} % Allows default copyright year (20XX) to be over-ridden
%- IF NEED BE.  \crdata{0-12345-67-8/90/01}  % Allows default copyright data
%(0-89791-88-6/97/05) to be over-ridden - IF NEED BE.  --- End of Author
%Metadata ---

\title{Summarizing Disasters Over Time}
%    Alternate {\ttlit ACM} SIG Proceedings Paper in LaTeX
%Format\titlenote{(Produces the permission block, and copyright information).
%For use with SIG-ALTERNATE.CLS. Supported by ACM.}} \subtitle{[Extended
%Abstract] \titlenote{A full version of this paper is available as
%\textit{Author's Guide to Preparing ACM SIG Proceedings Using
%\LaTeX$2_\epsilon$\ and BibTeX} at \texttt{www.acm.org/eaddress.htm}}}
%
% You need the command \numberofauthors to handle the 'placement and
% alignment' of the authors beneath the title.
%
% For aesthetic reasons, we recommend 'three authors at a time' i.e. three
% 'name/affiliation blocks' be placed beneath the title.
%
% NOTE: You are NOT restricted in how many 'rows' of "name/affiliations" may
% appear. We just ask that you restrict the number of 'columns' to three.
%
% Because of the available 'opening page real-estate' we ask you to refrain
% from putting more than six authors (two rows with three columns) beneath the
% article title.  More than six makes the first-page appear very cluttered
% indeed.
%
% Use the \alignauthor commands to handle the names and affiliations for an
% 'aesthetic maximum' of six authors.  Add names, affiliations, addresses for
% the seventh etc. author(s) as the argument for the \additionalauthors
% command.  These 'additional authors' will be output/set for you without
% further effort on your part as the last section in the body of your article
% BEFORE References or any Appendices.

\numberofauthors{3} %  in this sample file, there are a *total*
% of EIGHT authors. SIX appear on the 'first-page' (for formatting reasons)
% and the remaining two appear in the \additionalauthors section.
%
\author{
% You can go ahead and credit any number of authors here, e.g. one 'row of
% three' or two rows (consisting of one row of three and a second row of one,
% two or three).
%
% The command \alignauthor (no curly braces needed) should precede each author
% name, affiliation/snail-mail address and e-mail address. Additionally, tag
% each line of affiliation/address with \affaddr, and tag the e-mail address
% with \email.
%
% 1st. author
\alignauthor Chris Kedzie\\ \affaddr{Columbia University}\\
\affaddr{Department of Computer Science} \email{kedzie@cs.columbia.edu}
% 2nd. author
\alignauthor Kathleen McKeown\\ \affaddr{Columbia University}\\
\affaddr{Department of Computer Science}\\ \email{kathy@cs.columbia.edu}
% 3rd. author
\alignauthor Fernando Diaz\\ \affaddr{Microsoft Research}\\
\email{fdiaz@microsoft.com}
% There's nothing stopping you putting the seventh, eighth, etc.  author on
% the opening page (as the 'third row') but we ask, for aesthetic reasons that
% you place these 'additional authors' in the \additional authors block, viz.
} \date{30 July 1999}
% Just remember to make sure that the TOTAL number of authors is the number
% that will appear on the first page PLUS the number that will appear in the
% \additionalauthors section.

\maketitle \begin{abstract} We have developed a text summarization system that
can generate summaries over time from web crawls on disasters. We show that
our method of identifying exemplar sentences for a summary using affinity
propagation clustering produces better summaries than clustering based on
K-medoids as measured using Rouge on a small set of examples. A key component
of our approach is the prediction of salient information using event related
features based on location, temporal changes in topic, and two different
language models.  \end{abstract}

% A category with the (minimum) three required fields
\category{H.4}{Information Systems Applications}{Miscellaneous}
%A category including the fourth, optional field follows...
\category{D.2.8}{Software Engineering}{Metrics}[complexity measures,
performance measures]

\terms{Summarization}

\keywords{Extractive Summarization, Affinity Propagation}

\section{Introduction}

During crises, information is critical for first responders and those caught
in the event.  When the event is significant, as in the case of Hurricane
Sandy, the amount of information produced by traditional news outlets,
government agencies, relief organizations, and social media can vastly
overwhelm those trying to monitor the situation. Methods for identifying,
tracking, and summarizing events from text based input have been explored
extensively  (e.g.,
\cite{allan1998topic,Filatova&Hatzivassiloglou.04a,Wang&al.11}). However,
these experiments were not performed in the large and heterogeneous
environment of the modern web.



%Previous work on generating event descriptions and/or multi-document
%summarization has relied on clustering algorithms to find representative
%sentences appropriate for an event summary.  These methods impose a metric
%space on the text data that can make it difficult to incorporate external
%sources of information elegantly -- in this paper we argue that centroid
%sentences are not a priori the best candidates for inclusion in an event
%summary.

In this paper, we present an update summarization system to track events
across time. Our system predicts sentence salience in the context of a
large-scale event, such as a disaster,  and integrates these predictions into
a clustering based multi-document summarization system. We train a regression
model to predict sentence salience and use these predictions to bias the
formation of sentence clusters around more salient regions in the input space
using affinity propagation (AP) clustering.  AP uses the salience predictions
as well as pairwise similarities among input sentences to identify
\emph{exemplar} sentences, which we use as our summary output.  Our approach
differs from other methods of summarization that compute salience by pairwise
comparisons alone, ignoring features of importance that are intrinsic to the
sentences themselves.








%the formation of sentence clusters around more salient sentences. We
%introduce the affinity propagation algorithm as as an elegant way to
%incorporate our salience predictions into a

%What follows is a description of our ongoing event summarization efforts.  We
%briefly situate our approach to summarization within the broader field of
%multi-document summarization, and then introduce the affinity propagation
%algorithm which we use for clustering. This algorithm allows us to elegantly
%address the salient sentence selection problem by incorporating our prior
%beliefs about sentence quality. Next, we describe out method for modeling
%summary sentence quality, and the features used in this model.  Finally, we
%address future features and system improvements that we are incorporating
%into our summarizer.


\section{Related Work}

A principal concern in extractive multi-document summarization is the
selection of salient sentences for inclusion in summary output
\cite{nenkova2012survey}.  This has often been approached as a ranking
problem.
%We broadly conceptualize this decision as either an intrinsic or extrinsic
%sentence evaluation process. Intrinsic approaches evaluate sentences
%individually, possibly by predicting the impact on summary quality using
%sentece level features. 
Sentences have been ranked by the average word probability, average tf-idf
score, and the number of topically related words (topic-signatures in the
summarization literature)
\cite{nenkova2005impact,hovy1998automated,lin2000automated}. The first two
statistics are easily computable from the input sentences, while the third
only requires an additional, generic background corpus.  Another ranking
approach, centroid summarization, involves creating an average bag of words
(BOW) vector, the centroid, from the input sentences and ranking sentences by
their similarity to the centroid \cite{radev2004centroid}.  Graph
\cite{erkan2004lexrank} and clustering
\cite{hatzivassiloglou2001simfinder,mckeown1999towards,siddharthan2004syntactic}
based approaches, on the other hand, make use of pair-wise similarity
comparisons amongst input sentences.  In these models, salient sentences are
more central to the input or cluster, respectively.

%identify salient regions of the input space while simultaneously coping with
%redundancy.  Graph-based algorithms have been used to rank sentences
%Clustering algorithms, e.g., are commonly used to exploit redundancy in
%input. Input sentences are clustered and summaries are generated by selecting
%the most representative sentence from each cluster.  Graph-based models have
%also been used for summarization.  E.g., the LexRank algorithm treats
%sentences as nodes in a graph, where edges are constructed by way of cosine
%similarity between sentence nodes; edges are either continuosly weighted by
%similarity or discrete, existing only when the similarity is above a
%threshold.  The PageRank algorithm is used on the graph to find the most
%important sentence nodes. In both clustering and graph-based approaches,
%sentence salience is largely determined by the pairwise relations between
%sentences.

Supervised learning has also been applied to this task. Model features are
usually derived from human generated summaries, and are non-lexical in nature
(e.g., sentence starting position, number of topic-signatures, number of
unique words, word frequencies). Seminal work in this area has employed naive
Bayes and logistic regression classifiers to identify sentences for summary
inclusion \cite{kupiec1995trainable,conroy2001using}. 

%\fdadd{
Several researchers have recognized the importance of summarization during
natural disasters.  Guo \textit{et al.} developed a system for detecting
novel, relevant, and comprehensive sentences immediately after a natural
disaster \cite{qi:temporal-summarization}.  The method uses a model of
sentence relevance and novelty in order to select appropriate updates.
Training data for regression targets is automatically generated from
retrospective Wikipedia data.  The system is evaluated on news documents
related to 197 natural and human disasters from 2009 to 2011 using variants of
Rouge modified to capture novelty, relevance, and comprehensiveness
\cite{lin2004rouge}.  Wang and Li present a clustering-based approach to
efficiency detect important updates during natural disasters
\cite{wang:update-summarization}.  The algorithm works by hierarchically
clustering sentences online, allowing the system to output a more expressive
narrative structure than Guo \textit{et al.}.  The method is evaluated on
official press releases related to Hurricane Wilma  in 2005 using Rouge score
between the system summary and a manually generated target summary.
%}

This work uses the TREC Stream Corpus data set that is in use by the TREC
Temporal Summarization track \cite{frank2012building,aslam2013trec} .
Generally, last year's participants used a pipelined approach to build
summaries, generally ranking sentences, filtering out all but the most
relevant, and then performing some sort of deduplication/redundancy removal
step.  Ranking approaches ranged from simple query word match to more
sophisticated query expansion and and query based language model scoring
\cite{liu2013ictnet,xu2013trec,baruah2013univ}.  Perhaps most similar to our
approach, the system of \cite{xu2013trec} uses a weighted combination of
features (similarity to query, named entity frequency, predicate frequency,
presence of numerical values, sentence novelty, etc.) to score sentences;
sentences above a threshold are added to the summary.  Both the weights and
the threshold are selected by hand.


Our system seeks to combine the best of these approaches, using supervised
learning to predict salience rankings, and directly incorporate this
information in a clustering algorithm to bias the formation of sentence
clusters around highly salient regions. 

%Broadly, we conceive of the above 



%Another popular graph-based model




 
%Larger clusters generally indicate more important For example 





%\cite{} also explore the use of a hidden Markov model to predict summary
%inclusion based on previously selected sentences.


 
%ranking.
%
%The field of extractive summarization has long been focused on determing
%sentence importance or salience.  Many These efforts have largely involved
%unsupervised, data-driven approaches, where systems rank input sentences
%using features easily computable from the input, and sometimes an unrelated
%background corpus.  The most basic sentence rankings are by average word
%probability and average TF-IDF score, statistics that are computable from the
%input alone.  One of the more effective techniques, topic-sigantures,
%involves identifying the most topically related words in the input sentences.
%This is done using a log-likelihood ratio test ($\chi^2$) to determine which
%of two hypotheses hold for a word in the input: (1) the probability of the
%word in the input distribution and the background distribution are identical,
%or (2) the probability of the word is greater in the input distribution than
%in the background distribution. If (2) is true with some statistical
%significance, the word is said to be a topic signature. As with the previous
%features, a common approach is to rank sentences by the number of
%topic-signatures they contain.
%
%In mulit-document summarization, clustering algorthims are often employed to
%determine sentence importance, with larger clusters corresponding to more
%important topics. 
%
%Supervised learning approaches have also been explored. Features are usually
%derived from human generated summaries, and are non-lexical in nature (e.g.,
%sentence starting position, number of topic-signatures, number of unique
%words, word frequencies). Seminal work in this area has employed naive Bayes
%and logistic regression classifiers to identify sentences for summary
%inclusion. \cite{} also explore the use of a hidden Markov model to predict
%summary inclusion based on previously selected sentences.
%
%If this  
%
%tried to rank sentences by their overlap with query words. 
%


%and models are developed to predict the likelihood of a sentences inclusion
%in a summary. Naive Bayes, hidden Markov models, and logistic regression to
%predict a sentence's summary inclusion.
%
%Researches have also experimented with supervised learning for summarization,
%however, machine learning approaches have not yielded significant advances in
%generic summarization. One difficulty is that gold data is difficult to
%obtain.
%
%
%In  
%
%
%%which remain competetive to supervised machine-learning techniques.
%
%have ranked sentences by
%
%
%their average word probability, average TF-IDF weighting, and/or topic
%signatures; these representations are all easily computable from the input
%sentences, and, for topic signatures, an unrelated background corpus.  Topic
%signatures in
%
% 
%These efforts can be broadly split into two camps: feature based ranking and
%clustering.
%
%
%
%Determing sentence importance has been A large portion of automatic
%summarization research has focused on  
%
%

\section{Multi-Document Summarization Framework}

%To generate event summaries, we identify the most salient sentences from a
%collection of related documents and present these sentences as the event
%summary. This approach, commonly referred to as extractive summarization,
%avoids the complexity of parsing a semantic representation from the input
%sentences and subsequently generating a summary\cite{nenkova2012survey}.


A common approach to automatic summarization is to identify sentences with the
highest centrality with respect to the input sentences.  Intuitively,
sentences with a high degree of centrality are more semantically related to
the entire set of input sentences.  A summary can thus be obtained by
returning the $k$ most central sentences.  This generally implies the
calculation of pairwise distances between all sentences
\cite{radev2004centroid, erkan2004lexrank}.  In these approaches, sentences
are evaluated extrinsically by their distance to other sentences, either
directly \cite{erkan2004lexrank} or through an aggregate centroid object
\cite{radev2004centroid}. The distance between sentences is most commonly the
cosine distance of sentence term-vectors, but in general this can be an
arbitrary real-valued similarity function.
%; the distance function imposes a topology on the space of input sentences.

For some domains, it is very likely that we will have additional background
knowledge that could be predictive of sentence salience for the event being
summarized.  For example, certain kinds of information extracted from the
sentence text (e.g., temporal or geographic proximity) can indicate relevance
to a given event.  It would be difficult to incorporate this kind of salience
into the measure of centrality.


For example, consider a cluster of three sentence vectors $s_1 = (1,1,0)$,
$s_2 = (1,1,1)$, and $s_3 = (0,1,1)$. Without any other information, $s_2$ has
the highest degree of centrality, i.e. it has the highest average cosine
similarity and smallest average Euclidean distance to the other sentences.
Now, if we believe that $s_1$ is $\alpha$ times more salient than the other
sentences, we cannot simply scale $s_1$ by $\alpha$--the average cosine
similarity will remain unchanged, since the vector magnitude does not affect
the angle and $s_1$ will have an even greater the average Euclidean distance
from the rest.  Worse still, $s_2$ will still be the most representative
sentence of the three.

%For example, if we scaled the BOW vector based on our belief in its salience,
%it would not effect the cosine distance since the angle between two vectors
%is not changed by the magnitude of either

%Simply scaling the sentence vector based on our belief in it's fitness does
%not with our prior beliefs-- When using the BOWs representation, the input
%space is lexically defined, but our salience metric may not be based on any
%specific lexical items. 


%These approaches make it difficult to leverage prior knowledge to guide the
%selection of salient sentences. Sentences 


%
%Centroid based algorthims are commonly employed in the summarization
%literature to compress a large number of sentences into a much smaller,
%representative set. 
%
%The $k$-medoids clustering algorithm is commonly used to partition input
%sentences into topically distinct clusters, where the most representative
%sentences (the cluster medoids) are taken to build the summary. The quality
%of the summary in this framework is dependent on the method of sentence
%representation and the ability of a similarity or distance metric to
%adequately capture pairwise comparisons between sentences. Previous work in
%event clustering has examined the use of different term weightings (tf-idf,
%named-entities)  and lower dimensional term representations (LSA, LDA) for
%use in summarization.
%
%
%A popular graph-based alternative, LexRank, treats each sentence as a node in
%a graph, and identifies the nodes that are most central by in-degree.  The
%existence of an edge between two sentences is determined by a similarity
%function (either by setting a threshold above which the edge exists, or by
%taking the similarity as a continuous edge weight). The sentences with the
%highest eigenvector centrality can be found using random walks (PageRank).
%The $k$ most central sentences are used as the summary.
%
%One drawback of these algorithms is that they do not provide an elegant way
%to incorporate prior information about sentence salience.

In our approach, the system generates clusters using an affinity propagation
algorithm and from each cluster an exemplar sentence is selected that is added
to the summary.  In the following sections, we show how  prior information
representing salience can easily be incorporated into the affinity propagation
algorithm.  We believe the incorporation of salience to be useful in noisy
environments (e.g., a web crawl), and that it can help the formation of
clusters around the most relevant inputs. Our current system  is trained using
features derived from location, changes in wording across time and language
models that characterize the language of disaster to generate summaries at
regular intervals across time.  As we develop the system further, we will
extend it to generate updates across time,  penalizing the salience of
concepts already selected by the summarizer to encourage the discovery of
novel sentences as the event unfolds. 



\subsection{Data}

Our documents for summarization come from the online news portion of the TREC
Stream Corpus, a 6.45tb corpus obtained by hourly web crawls from October 2011
through mid February 2013
\cite{frank2012building}.\footnote{\url{http://trec-kba.org/kba-stream-corpus-2014.shtml}}
Summary events come from the TREC Temporal Summarization track, and include
natural disasters like Hurricane Sandy as well as man-made events like a 2012
train accident in Buenos
Aires.\footnote{\url{http://trec.nist.gov/data/tempsumm2013.html}} The track
organizers also provide a search query for each event \cite{aslam2013trec}.
For each event, we collect the documents that contain all query words and
stratify them by the hour they were collected. 

For evaluation purposes, the track organizers also provided gold nugget
information (i.e. important pieces of information, usually the length of a
short clause or sentence). These gold nuggets come from the event's related
Wikipedia article and also include the timestamp of when they were added to
the page.

%KM _ Ask Chris: is this true?
To create hourly gold summaries to evaluate our system, we simply take the set
of gold nugget information from the start of the event up to the current hour. 

\subsection{Affinity Propagation}

Affinity propagation (AP) is a message passing algorithm that identifies both
exemplar data points and assignments of each point to an exemplar.  This is
done iteratively by passing \emph{responsibility} and \emph{availability}
messages between data points that quantify the fitness of one data point to
represent another, and the fitness of a data point to be represented based on
the choices of other data points respectively \cite{dueck2007non}.


%Each element $R(i,k)$ expresses the fitness of the $k^{th}$ point to serve as
%the exemplar of the $i^{th}$ point relative to other potential exemplars.
%Each element $A(i, k)$ represents the $k^{th}$ point's ``availability'' to
%serve as an exemplar of the $i^{th}$ point, taking into account other points'
%preference for the $k^{th}$ point as an exemplar.

AP is parameterized by a $n\times n$ similarity matrix $S$ and a $n\times 1$
preference vector $\pi$.  $S$ is a real-valued matrix where $S(i,j)$ is the
similarity of the $i$-th data point to the $j$-th data point.  $S$ does not
need to be symmetric.  $\pi$ is a real-valued vector where $\pi(i)$ expresses
our preference that the $i$-th data point can serve as an exemplar a priori of
other data points. 

In our experiments $\pi(i)$ is set to the salience prediction from the
Gaussian process regression for the $i$-th sentence minus an offset. This
offset drives most of the preferences negative and reduces the number of
returned exemplars to a handful of sentences (around 4-5).  For the similarity
matrix, we use $S(i,j) = -\operatorname{dist}(i,j)^2$, where
$\operatorname{dist}$ is the Euclidean distance between the BOW vectors for
sentences $i$ and $j$.
%we used the negative squared cosine distance of sentence bag-of-words. 
The summary output is the set $\mathcal{E}$ of returned exemplars found after
convergence.




%While, AP does not set the number of exemplars before-hand, a lower overall
%preference values will result in a smaller number of exemplars.


%arbitrary pair-wise similarity function $S: \mathcal{R}^d \rightarrow
%\mathcal{R}$, where $d$ is the dimension of the data being clustered and a
%real-valued preference $\pi_i$ quantifying our belief a priori of the
%$i^{th}$ element's ability to serve as an exemplar.

AP has two useful properties for summarization. First, the number of clusters
identified is determined by the preferences -- lower overall preference
values  will result in fewer clusters. Unlike $k$-means, we do not have to
specify how many clusters we would like to find.  Determining the number of
clusters in a principled way each time we run the clustering algorithm would
be difficult in our setup.  Secondly, the arbitrary nature of the preferences
and similarity function allow us to incorporate a variety of signals for
identifying the best exemplars. 


\subsection{Predicting Sentence Salience}

In order to use AP clustering for summarization, we need to assign a
preference value to each input sentence.  In our approach, we equate a
sentence's salience with its preference.  A good model of sentence salience
should predict higher values for sentences that are more likely to appear in a
human generated summary of the event.

To build training data for this regression task, we take a subset of sentences
relevant to the TREC events (approximately 1000) and match them to the gold
nugget sentence with highest similarity as determined by the sentence
similarity system of \cite{guo2012simple}. 
%\cite{} have used this system previously to correlate sentences to meaningful
%units of information in human generated summaries. 
We use the real-valued similarity scores as our salience scores for the
training sentences.

We want our model to be predictive across different kinds of events so we
avoid lexical features.  Instead, we extract a variety of features including
language model scores, geographic relevance, and temporal relevance from each
sentence.  These features are used to fit a Gaussian process regression model
that can predict the similarity of a sentence to a gold summary
\cite{preotiuc2013temporal}.  We use the model predicted salience of each
sentence as it's preference value in the AP clustering. 
%In order to predict the salience score for a sentence, 

 %\cite{guo2012simple} produces a real-valued score of the similarity We use
 %\cite{guo2012simple} produces real valued similarity scores s.
 %\cite{guo2012simple} determines the similarity of two sentences by deriving
 %low dimensional We have As a proxy measure for sentence salience, we use the
 %maximum similarity of an input sentence to a human generated summary
 %sentence, using the state-of-the-art sentence similarity module of
 %\cite{guo2012simple}. \cite{guo2012simple} derive a lower-dimensional vector
 %of latent features for a sentence. The similarity of two sentences is then
 %the cosine similarity of the latent vectors.  In order to predict this
 %similarity we extract a variety of features including language model scores,
 %geographic relevance, and temporal relevance.  We learn a predictive
 %function of sentence-summary similarity by fitting a Gaussian process model
 %on these features. \cite{preotiuc2013temporal}

%As training data for this model, we take a portion of the sentences found in
%the relevant documents and find their highest similarity to the gold nugget
%information seen thus far.  

%When assigning preference scores


%After obtaining similarity scores for sentences in our training set, we fit a
%Gaussian process model to learn 


%Our model 




%Correllating automatic measures of sentence salience with human judgments is
%a difficult task. , we use the maximum cosine similarity to 

\begin{table*}[ht] \begin{center} \begin{tabular}{| c p{7.0cm} || c p{7.0cm}
|} \hline Preference & AP Clustering & Preference & $k$-Medoids Clustering\\
\hline 9.010 & The magnitude-7.5 quake, about 20 miles deep, was centered off
the town of Champerico .People fled buildings in Guatemala City , in Mexico
City and in the capital of the Mexican state of Chiapas , across the border
from Guatemala & 9.010 & The magnitude-7.5 quake, about 20 miles deep, was
centered off the town of Champerico .People fled buildings in Guatemala City ,
in Mexico City and in the capital of the Mexican state of Chiapas , across the
border from Guatemala   \\ \hline 9.010 & A reporter in the town of San Marcos
, about 80 miles north of the epicenter, told local radio station Emisoras
Unidas that houses had collapsed onto residents and smashed televisions and
other appliances had been scattered into the streets.  & 3.007 & ``Things fell
in my kitchen .'' Perez said more than 2,000 soldiers were deployed from a
base in San Marcos to help with disaster relief.\\ \hline 9.007 & The local
fire department said on its Twitter account that a school had collapsed and
eight injured people had been taken to a nearby hospital.  & 3.007 &  Ingrid
Lopez , who went to the hospital with a 72-year-old aunt whose legs was
crushed by a falling wall, said she had waited hours for an X-ray.\\ \hline
7.008 & There are three confirmed dead and many missing after the strongest
earthquake to hit Guatemala since a deadly 1976 quake that killed 23,000.  &
1.007 & Hundreds of people crammed into the hallways of the small town
hospital waiting for medical staff to help out hundreds of injured family
members, some complaining they were not getting care quickly enough. \\ \hline
\end{tabular} \end{center} \caption{Example summary using affinity propagation
(left) and $k$-medoids (right)} \end{table*} \begin{table*}[t] \begin{center}
\begin{tabular}{| c || c | c | c || c | c | c || c | c | c |} \hline Method &
\multicolumn{3}{c}{ROUGE-1} & \multicolumn{3}{c}{ROUGE-2} &
\multicolumn{3}{c}{ROUGE-3} \\ & Recall & Prec. & F-1 & Recall & Prec. & F-1 &
Recall & Prec. & F-1 \\ \hline $k$-medoids & 0.127 & 0.414 & 0.181 & 0.025 &
0.076 & 0.035 & 0.003 & 0.010 & 0.005\\ \hline AP & 0.117 & 0.440 & 0.173 &
0.022 & 0.082 & 0.033 & 0.004 & 0.015 & 0.006\\ \hline \end{tabular}
\end{center} \caption{ROUGE scores for $k$-medoids and affinity propagation
methods} \end{table*}



\subsection{Basic Features}

We employ several basic features that have been used previously in supervised
models to rank sentence salience \cite{kupiec1995trainable,conroy2001using}.
These include sentence length, the number of capitalized words normalized by
sentence length, and the number of query words present in the sentence.  Query
words include the event's type (e.g., \emph{earthquake}) and are expanded with
the event type's WordNet \cite{miller1995wordnet} synset, hypernyms, and
hyponyms.  For \emph{earthquake}, e.g., we obtain ``quake,'' ``temblor,''
``seism,'' ``aftershock,'' etc.   


\subsection{Language Model Features}

%Because the data in our experiments is scraped from the web, it is common to
%find sentences that contain both salient informantion and two kinds of noise:
%noisey fragments of web page structure (e.g. section titles, \emph{News},
%\emph{Sports}, etc.) and references to other news not relevant to the topic
%summary.
%
We use two trigram language models, trained using the SRILM toolkit
\cite{stolcke2002srilm}, taking as features the average log probability (i.e.
the sentence's total log probability normalized by sentence length) from each
model.  This first model is trained on 4 years (2005-2009) of articles from
the Gigaword corpus.  Specifically, we use articles from the Associated Press
and the New York Times. This model is intended to assess the general writing
quality (grammaticality, word usage) of an input sentence and helps us to
filter out text snippets which are not sentences (e.g., web page titles).  The
second model is a domain specific language model. We build a corpus of
Wikipedia articles for each event type, consisting of documents from a related
Wikipedia category. E.g. for earthquakes, we collect pages under the category
\emph{Category:Earthquakes}. This model assigns higher probability to
sentences that are focused on the given domain.

%For both models, we Finally, we extract the percentage of capitalized words,
%and sentence length as features. These last two features also help to
%identify sentences that are less likely to contain relevant content-- overly
%long and heavily capitalized sentences in our corpus were likely to be long
%strings of web-page headlines, section headers, and other irrelevant page
%structure. 

\subsection{Geographic Relevance Features}

Locations are identified using a named entity tagger. For each location in a
sentence, we obtain its latitude and longitude using the Google Maps API.  We
then compute its distance to that of the event location.  It is possible for a
sentence and an event to have multiple locations so we take as features the
minimum, maximum, and average distance of all sentence-event location pairs.
Distances are calculated using the Vincenty distance.

\subsection{Temporal Relevance Features}

Our data consists of hourly crawls of online content and so we exploit the
temporality of corpus by capturing the burstiness of a sentence, i.e.  the
change in word frequency from one hour to the next.``Bursty'' sentences often
indicate new and important data. 

Let $D_t$ be the set of web pages at time $t$ and let $s = \{w_1,\ldots,
w_n\}$ be a sentence from a page $d \in D_t$.  We calculate the 1-hour
burstiness of sentence $s$ from document $d$ at hour $t$  as \begin{align*}
\operatorname{b}_1(s,d,t) = \frac{1}{|s|} \sum_{w \in s} \Bigg( &
\operatorname{tf-idf}_t(w,d)  \\ & \left. - \frac{\sum_{d^\prime \in D_{t-1}:
w \in d^\prime } \operatorname{tf-idf}_{t-1}(w,d^\prime)}{|\{d^\prime \in
D_{t-1}: w \in d^\prime\}|} \right) \end{align*}

where \begin{align*} \operatorname{tf-idf}_t(w,d) =&
\log\left(1+\sum_{w^\prime \in d}1\{w=w^\prime\}  \right)\\ & \times
\log\left(\frac{|D_t|}{1 + \sum_{d^\prime \in D_t}1\{w \in d^\prime\}}\right).
\end{align*}
% 1\{w = w^\prime} %- \operatorname{avg-tf-idf}_{t_{i-1}}(w).
%\end{align*}


We similarly find the sentence's 5-hour burstiness.  In addition to
burstiness, we also include the sentence's average tf-idf and hours since the
event in question started as features.


\section{Experiments}

We carried out a small set of initial experiments on one event.  We collected
a subset of pages from the TREC Stream Corpus that were relevant to a 2012
earthquake off the coast of Guatemala, and further subdivided this collection
by the hour they were created. For each hour we generated a summary using the
AP clustering algorithm. 

 We also generated baseline summaries using the $k$-medoids (using the
 Partitioning Around Medoids algorithm), setting $k = |\mathcal{E}|$ , i.e.
 the number of exemplar sentences returned by the AP. Because $k$-medoids
 begins with a random initialization, we took the best (minimum average
 distance) result of 100 restarts.

Table 1 shows example output of the AP and $k$-medoids generated summaries.
Sentences are ranked by preference score, although preference has no effect on
the $k$-medoids algorithm.  Quantitatively, AP exemplar sentences had higher
predicted sentence quality scores (preferences) than the cluster medoids.
Qualitatively, the AP method appears to select more general details about the
earthquake. Looking at the third sentence selection in table 1, we can see
that $k$-medoids selects a personal experience that was reported. This is
perhaps less newsworthy or reportable compared to the third sentence in in the
AP generated summary which reports a notable structure collapse and injuries
related to the quake.  We believe AP results in a more readable and
informative summary, although we have yet to perform a rigorous human
evaluation of the summary output.


We evaluated both algorithms with the ROUGE toolkit \cite{lin2004rouge}.
N-ROUGE works by calculating the n-gram recall and precision of an
automatically generated summary in reference to a model summary.  We created
model summaries by taking the gold nugget sentences with timestamps up to and
including the current system time as the gold summary for that hour.
%\fdcomment{may be interesting to measure track metrics to measure latency,
%redundancy, etc.} KM - Ask Chris

Table 2 shows average recall, precision, and F-measure for various orders of
ROUGE score. AP demonstrated consistently higher precision than our baseline.
%scores over $k$-medoids (ROUGE evaluations are usually focused on recall).
While not statistically significant, it is difficult to show significance with
Rouge using a small test; we hope further tests will confirm this improvement. 
%   , AP had higher precision on average.
On average, the AP summaries were slightly shorter than the baseline, which
would partially explain this difference. It is also possible that our language
models are biased toward shorter sentences; we are more likely to have seen a
shorter sentence in the language model input.
%\fdcomment{is there a theory to why this is happening?}
We are currently adapting our summarizer to add updates over time, and
maintaining precision will be important to prevent topic drift.  



%\begin{table}[h!] \tiny \begin{center} \begin{tabular}{| c p{3.5cm} |} \hline
%Preference & Sentence\\ \hline 9.010 & The magnitude-7.5 quake, about 20
%miles deep, was centered off the town of Champerico .People fled buildings in
%Guatemala City , in Mexico City and in the capital of the Mexican state of
%Chiapas , across the border from Guatemala\\ \hline 3.007 & "Things fell in
%my kitchen ." Perez said more than 2,000 soldiers were deployed from a base
%in San Marcos to help with disaster relief.\\ \hline 3.007 & Ingrid Lopez ,
%who went to the hospital with a 72-year-old aunt whose legs was crushed by a
%falling wall, said she had waited hours for an X-ray. \\ \hline 1.007 &
%Hundreds of people crammed into the hallways of the small town hospital
%waiting for medical staff to help out hundreds of injured family members,
%some complaining they were not getting care quickly enough. \\ \hline
%\end{tabular} \end{center} %  \caption{Example summary using $k$-medoids with
%100 random restarts} \end{table}




\section{Conclusions and Future Work}

We have developed a summarizer that can generate summaries over time from web
crawls on disasters. We show that our method of identifying exemplar sentences
for a summary using AP clustering produces  summaries with higher precision
    compared to those based on clustering with K-medoids. A key component of
    our approach is the prediction of salient information using features based
    on location, temporal changes in topic, and two different language models.

Currently, we run each hour of summarization independently. In order to avoid
repeating information, we would like to incorporate previously chosen
exemplars in the preference computation. One possibility would be to
down-weight a candidate exemplar's preference based on its similarity to
previous exemplars.

Secondly, we would like to do more intelligent inference of missing
geographical information since not all sentences contain locations. Currently
we are using mean values for missing data.

Finally, we would like to experiment with non-symmetric similarity matrices,
specifically using narrative chains\cite{chambers2009unsupervised}. Under this
model $S(i,j)$ would express the likelihood that the events in sentence $j$
precede the events in sentence $i$. We hope such a parameterization would
promote more causally motivated sentences into exemplar positions, which would
better describe the disaster event domain.

%
\bibliographystyle{abbrv}
%\bibliography{sigproc}  % sigproc.bib is the name of the Bibliography in this
%case
\bibliography{kdd.bib}
% You must have a proper ".bib" file and remember to run: latex bibtex latex
% latex to resolve all references
%
% ACM needs 'a single self-contained file'!
%
%APPENDICES are optional \balancecolumns \balancecolumns % GM June 2007 That's
%all folks!
\end{document}
