\section{Problem Definition}
\label{sec:problemdefinition}
In order to evaluate a temporal summarization system, we adopt the simulation-based approach used in the TREC Temporal Summarization Track.  We provide a brief overview of the problem.  Details on the formulation can be found in the track overview paper \cite{aslam2013trec}.  

A temporal summarization system takes as input 
\begin{enumerate*}[label=\itshape\alph*\upshape)]
  \item a short query $\query$ defining the event to be tracked (e.g. `Hurricane Sandy'), 
  \item an event category $\eventcategory$ defining the type of event to be tracked (e.g. `hurricane'), 
  \item a stream of time-stamped documents, $(\doc_0, \doc_1,\ldots,\doc_T)$, presented in temporal order, and
        \item an evaluation time period of interest, $(\stime,\etime)$.  
\end{enumerate*}
While processing documents throughout the time period of interest, the system
must output sentences, known as updates, which are \emph{relevant} to the
query, \emph{comprehensive} with respect to the event, \emph{novel} to the
user, and \emph{timely} with respect to when the update occurred.  Precisely
how to measure these properties will be discussed in Section \ref{sec:methods}.
In our version of this problem, we assume that that the system receives one
batch of new sentence-segmented documents every hour throughout the period of
interest.

We address  disaster types such as terrorism and mass shootings (e.g.,
the 2012 shooting in Aurora, Col.), natural disasters (e.g., Hurricane Sandy),
accidents (e.g., the 2012 Pakistan garment factory fire), astronomical
disasters (e.g,. the Chelyabinsk comet in Russia), and social activism (e.g.,
the Arab spring).
\kmcomment{This was as best as I could do on this. These are the example
  categories from the wikipedia page. However you refer in the language model
  description to the event type earthquake which is more specific. Also, in the
  TREC data you refer to event types that are specified in the metadata. So I'm
  wondering how many different categories you have and where they come from.}

\fdcomment{include description of nuggets here.}
\kmcomment{We should jointly discuss what goes in data and what goes in problem
  definition. I have moved event types here because I think knowing how many
  types and examples of them would be helpful upfront. I think where nuggets
  and wikipedia pages go is questionable.}
