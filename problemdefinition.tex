\section{Problem Definition}
\label{sec:problemdefinition}
In order to evaluate a temporal summarization system, we adopt the simulation-based approach used in the TREC Temporal Summarization Track.  We provide a brief overview of the problem.  Details on the formulation can be found in the track overview paper \cite{aslam2013trec}.  

A temporal summarization system takes as input 
\begin{enumerate*}[label=\itshape\alph*\upshape)]
  \item a short query $\query$ defining the event to be tracked (e.g. `Hurricane Sandy'), 
  \item an event category $\eventcategory$ defining the type of event to be tracked (e.g. `hurricane'), 
  \item a stream of time-stamped documents, $(\doc_0, \doc_1,\ldots,\doc_T)$, presented in temporal order, and
	\item an evaluation time period of interest, $(\stime,\etime)$.  
\end{enumerate*}
While processing documents throughout the time period of interest, the system must output sentences, known as updates, which are \emph{relevant} to the query, \emph{comprehensive} with respect to the event, \emph{novel} to the user, and \emph{timely} with respect to when the update occurred.  Precisely how to measure these properties will be discussed in Section \ref{sec:methods}.  In our version of this problem, we assume that that the system receives one batch of new sentence-segmented documents every hour throughout the period of interest.  

\fdcomment{include description of nuggets here.}