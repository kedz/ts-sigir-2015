\section{Problem Definition}
\label{sec:problemdefinition}
In order to evaluate a temporal summarization system, we adopt the simulation-based approach used in the TREC Temporal Summarization Track.  We provide a brief overview of the problem.  Details on the formulation can be found in the track overview paper \cite{aslam2013trec}.  

A temporal summarization system takes as input 
\begin{enumerate*}[label=\itshape\alph*\upshape)]
  \item a short query $\query$ defining the event to be tracked (e.g. `Hurricane Sandy'), 
  \item a stream of time-stamped documents, $(\doc_0, \doc_1,\ldots,\doc_T)$, presented in temporal order, and
	\item an evaluation time period of interest, $(\stime,\etime)$.  
\end{enumerate*}
Throughout the time period of interest, the system must output sentences, known as \emph{updates}, which are \emph{relevant} to the query, \emph{comprehensive} with respect to the event, \emph{novel} to the user, and \emph{timely} with respect to when the update occurred.  We present the evaluation metrics which measure these properties in Section \ref{sec:methods}.  

In the ideal version of this problem, the system makes decisions every time it consumes a document $\doc_i$.  In practice, we assume that that we receive batches of new sentence-segmented documents every hour.  
