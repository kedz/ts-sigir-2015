\section{Motivation}

Most multi-document summarization approaches involve either sentence 
clustering or ranking or both to select summary sentences.
When the summarization task
is broad or underspecified, clustering is most appropriate in that the data
``speaks for itself.'' This is probably most true when summarization is 
interpreted to mean exploratory data analysis. 
When there are more constraints on the output, as in many text summarization
tasks,
it is often easier to design appropriate features for the ranking approach, 
either by rule or learning-to-rank. The primary contribution of this paper
is a framework for combining both of these approaches using affinity
propagation clustering. Additionally, we provide a task specific evaluation
of this clustering model on the TREC Temporal Summarization task. 


\ckcomment{Not sure how to present this together, we also essentially evaluate
a salience model by feature ablation for the TS task, but this feels very specific compared to the general framework idea.}

\fdcomment{I guess I think we need to two things in this section (and/or introduction).  1. explain why the problem matters.  2. explain why our approach is compelling.  }

\fdcomment{if easy to generate, would be good to have a single figure for an event that includes 1. the query, 2. the distribution of (filtered) document volume over time, and 3. a sample of the gold nuggets.  this provides the reader with a visual explanation of the problem, the solution, and the difficulty}