\label{sec:introduction}
During crises, information is critical for first responders,
 crisis management organizations, and those caught in the event.  
When the event is significant, as in the case of Hurricane
Sandy, the amount of content produced by traditional news outlets,
government agencies, relief organizations, and social media can vastly
overwhelm those trying to monitor the situation. 
Crisis informatics \cite{palen2010vision} is dedicated to finding methods for sharing the
right information in a timely fashion during such an event.
Research in this field has focused on
human-in-the-loop approaches ranging from on the ground information 
gathering to crowdsourced reporting and disaster management \cite{starbird2013working}.

Multi-document summarization has the potential to assist the crisis 
informatics community. 
Automatic summarization could deliver
relevant and salient information at regular intervals, 
even when human volunteers are unable to. 
Perhaps more importantly it could help filter out
unnecessary and irrelevant detail when the volume of incoming information
is large. 
While methods for identifying,
tracking, and summarizing events from text based input have been explored
extensively
%KM - I think we should consider adding other event summarizers. 
\cite{allan1998topic,Filatova&Hatzivassiloglou.04a,Wang&al.11}, 
these experiments were not developed to handle streaming data from a
 heterogeneous environment at web scale. 
These methods also rely heavily on redundancy which is suboptimal for time
sensitive domains where there is a high cost in delaying information.

In this paper, we present an update summarization system to track events
across time. Our system predicts sentence salience in the context of a
large-scale event, such as a disaster, and integrates these predictions into
a clustering based multi-document summarization system. 
We demonstrate that combining salience with clustering produces more relevant summaries
compared to baselines using clustering or relevance alone.  
Our experiments suggest that this
is because our system is better able to adapt to
dynamic changes in input volume which adversely 
affect methods that use redundancy as a proxy for salience. 
%or predict salience but
%ignore redundancy in the input alone.


In addition to the tight integration between clustering and salience
prediction, our approach also exploits knowledge about the event to determine
salience. Thus, salience represents both how typical a sentence is of the  
event
type (e.g., industrial accident, hurricane, riot) and whether it specifies 
information
about this particular event. 
Our feature representation includes a set of language models, one for each
event type, to measure the typicality of the sentence with regard to the 
current event, the distance of mentioned locations from the center of
the event, and the change in word frequencies over the time of the event.
While we evaluate these features in the domain of disaster, this approach is generally applicable to many update summarization tasks.
% where queries can be categorized by the 
%expected similarity of their ideal summaries.
%While we ground these features in the domain of disaster, this approach is generally applicable to any update summarization task where queries can be categorized by the 
%expected similarity of their ideal summaries.
% to represent its likelihood of referring to the input disaster. 
%We also utilize the change in word frequencies as we process the stream 
%to measure the changing importance of updates over time.

Our approach achieves a statistically significant improvement in 
 ROUGE scores compared to multiple baselines.
Additionally, we introduce novel methods for estimating the average information
gain each update provides and how completely the update summary covers 
the event it is tracking; our system's updates contain more relevant
information on average than the competeing baselines.


The remainder of the paper is organized as follows.
We begin with a review of related work
in the information retrieval and multi-document
summarization literature. Section~\ref{sec:methods} outlines the details
of our salience and summarization models. Next we describe our
data (Section~\ref{sec:data}) and experiments (Section~\ref{sec:exper}). Finally,
we discuss our results (Section~\ref{sec:results}) and conclude the paper.

%Our main contributions are three fold. First, we present a straightforward 
%framework for 
%combining prior beliefs with a clustering algorithm. Second, we demonstrate
%the empirical effectiveness of this approach on the update summarization
%task. Finally, we present a novel feature represention for sentence salience
%in the crisis/disaster domain.



