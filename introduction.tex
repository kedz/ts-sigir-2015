\section{Introduction}
\label{sec:introduction}
During crises, information is critical for first responders and those caught
in the event.  When the event is significant, as in the case of Hurricane
Sandy, the amount of information produced by traditional news outlets,
government agencies, relief organizations, and social media can vastly
overwhelm those trying to monitor the situation. Methods for identifying,
tracking, and summarizing events from text based input have been explored
extensively
%KM - I think we should consider adding other event summarizers. 
(e.g.,
\cite{allan1998topic,Filatova&Hatzivassiloglou.04a,Wang&al.11}). However,
these experiments were not developed to handle streaming data from  the large and heterogeneous
environment of the modern web. Neither were they developed to address the
information needs that arise during  crisis
situations. 
%KM - this last sentence sounds somewhat negative.
%there is still a need for robust and scalable 
%methods for automatic summarization.

In this paper, we present an update summarization system to track disasters
across time. Our system predicts sentence salience in the context of a
large-scale event, such as a disaster,  and integrates these predictions into
a clustering based multi-document summarization system. We train a regression
model to predict sentence salience and use these predictions to bias the
formation of sentence clusters around more salient regions in the input space
using affinity propagation (AP) clustering.  AP uses the salience predictions
as well as pairwise similarities among input sentences to identify
\emph{exemplar} sentences, which we use as our summary output.  Our approach
differs from other methods of summarization that compute salience by pairwise
comparisons alone, ignoring features of importance that are intrinsic to the
sentences themselves.

%KM - Chris - add a short overview on the kinds of things we measure and
%results so people are encouraged to read ahead. Keep it general.

In addition to the tight integration between clustering and salience
prediction, our approach also exploits knowledge about disaster to determine
salience. Thus, salience does not just represent importance within a set of
documents; it also represents both how typical this sentence is of the input event
type (i.e., disaster, hurricane, tornado) and whether it specifies information
about this particular disaster. We use a set of language models, one for each
disaster type, to measure typicality of the sentence for the current event type. We
use a feature that measures distance of mentioned location from the center of
the disaster to represent its likelihood of referring to the input disaster. 




%Previous work on generating event descriptions and/or multi-document
%summarization has relied on clustering algorithms to find representative
%sentences appropriate for an event summary.  These methods impose a metric
%space on the text data that can make it difficult to incorporate external
%sources of information elegantly -- in this paper we argue that centroid
%sentences are not a priori the best candidates for inclusion in an event
%summary.



The remainder of the paper is organized as follows. We begin with a review of related work in the information retrieval and multi-document summarization literature.  In section
~\ref{sec:background} we give an overview of the semantic similarity,
%KM - Chris can you update this so it reflects the contributions. We don't
%mention semantic similarity. We do mention salience and the disaster specific
%features. Perhaps we should mention the use of a new approach for semantic
%similarity? 
Gaussian process, and affinity propagation algorithms that make up the bulk of
our TS system. 
Next we describe the data with which we build our various models.
Then in section~\ref{sec:approach} we give a high level sketch
of our system, and then explain each component in detail. 
