% This is "sig-alternate.tex" V2.0 May 2012 This file should be compiled with
% V2.5 of "sig-alternate.cls" May 2012
%
% This example file demonstrates the use of the 'sig-alternate.cls' V2.5
% LaTeX2e document class file. It is for those submitting articles to ACM
% Conference Proceedings WHO DO NOT WISH TO STRICTLY ADHERE TO THE SIGS
% (PUBS-BOARD-ENDORSED) STYLE.  The 'sig-alternate.cls' file will produce a
% similar-looking, albeit, 'tighter' paper resulting in, invariably, fewer
% pages.
%
% ----------------------------------------------------------------------------------------------------------------
% This .tex file (and associated .cls V2.5) produces: 1) The Permission
% Statement 2) The Conference (location) Info information 3) The Copyright
% Line with ACM data 4) NO page numbers
%
% as against the acm_proc_article-sp.cls file which DOES NOT produce 1) thru'
% 3) above.
%
% Using 'sig-alternate.cls' you have control, however, from within the source
% .tex file, over both the CopyrightYear (defaulted to 200X) and the ACM
% Copyright Data (defaulted to X-XXXXX-XX-X/XX/XX).  e.g.
% \CopyrightYear{2007} will cause 2007 to appear in the copyright line.
% \crdata{0-12345-67-8/90/12} will cause 0-12345-67-8/90/12 to appear in the
% copyright line.
%
% ---------------------------------------------------------------------------------------------------------------
% This .tex source is an example which *does* use the .bib file (from which
% the .bbl file % is produced).  REMEMBER HOWEVER: After having produced the
% .bbl file, and prior to final submission, you *NEED* to 'insert' your .bbl
% file into your source .tex file so as to provide ONE 'self-contained' source
% file.
%
% ================= IF YOU HAVE QUESTIONS ======================= Questions
% regarding the SIGS styles, SIGS policies and procedures, Conferences etc.
% should be sent to Adrienne Griscti (griscti@acm.org)
%
% Technical questions _only_ to Gerald Murray (murray@hq.acm.org)
% ===============================================================
%
% For tracking purposes - this is V2.0 - May 2012

\documentclass{sig-alternate} 
\usepackage{url} 
\usepackage{color}
% \documentclass[10pt]{article} \usepackage{url} \usepackage{color}
\usepackage[utf8]{inputenc}
% \usepackage[margin=1in]{geometry}
\usepackage{amssymb}
\usepackage{amsmath}
\usepackage[inline]{enumitem}
\usepackage{amsfonts}
\usepackage[]{algorithm2e}
\usepackage{cleveref}
\crefname{section}{§}{§§}
\Crefname{section}{§}{§§}
\usepackage{graphicx}
\usepackage{authblk}
\newcommand{\fdadd}[1]{\textcolor{red}{#1}}
\newcommand{\fdcomment}[1]{\textbf{\textcolor{red}{[FD: #1]}}}
\newcommand{\ckcomment}[1]{\textbf{\textcolor{blue}{[CK: #1]}}}
\newcommand{\kmcomment}[1]{\textbf{\textcolor{green}{[KM: #1]}}}
\DeclareMathOperator{\eventcategory}{c}
\DeclareMathOperator{\corpus}{\mathcal{C}}
\DeclareMathOperator{\doc}{\mathnormal{d}}
\DeclareMathOperator{\sent}{\mathnormal{s}}
\DeclareMathOperator{\order}{\pi}
\DeclareMathOperator{\dtime}{\mathnormal{t}}
\DeclareMathOperator{\hour}{\mathnormal{h}}
\DeclareMathOperator{\hours}{\mathcal{H}}
\DeclareMathOperator{\Sim}{\mathbf{K}}
\DeclareMathOperator{\SMat}{\mathbf{X}}
\DeclareMathOperator{\Pref}{\boldsymbol{\pi}}
\DeclareMathOperator{\Exemp}{\mathnormal{Exemplars}}
\DeclareMathOperator{\Updates}{\mathnormal{Updates}}


\newcommand{\tsvec}[1]{\mathbf{#1}}
\newcommand{\tsveci}[2]{#1_{#2}}
\newcommand{\tsmat}[1]{\mathbf{#1}}
\newcommand{\tsmatij}[3]{#1_{#2#3}}

\newcommand{\query}[0]{q}
\newcommand{\stime}[0]{t_s}
\newcommand{\etime}[0]{t_e}

\newcommand{\features}[0]{\tsvec{x}}
\newcommand{\featuresi}[1]{\tsveci{x}{#1}}

\newcommand{\kernelMatrix}[0]{\tsmat{K}}
\newcommand{\kernelMatrixij}[2]{\tsmatij{K}{#1}{#2}}

\newcommand{\numfeatures}[0]{n}

\newcommand{\candidates}[0]{\mathcal{S}}
\newcommand{\numcandidates}[0]{|\candidates|}

\newcommand{\candidateSimMat}[0]{\tsmat{S}}
\newcommand{\candidateSimMatij}[2]{\tsmatij{S}{#1}{#2}}

\newcommand{\preferences}[0]{\tsvec{\pi}}

\begin{document}

%
% --- Author Metadata here ---
%\conferenceinfo{WOODSTOCK}{'97 El Paso, Texas USA}
%\CopyrightYear{2007} % Allows default copyright year (20XX) to be over-ridden
%- IF NEED BE.  \crdata{0-12345-67-8/90/01}  % Allows default copyright data
%(0-89791-88-6/97/05) to be over-ridden - IF NEED BE.  --- End of Author
%Metadata ---

\title{The Anatomy of a Temporal Summarization System}

\numberofauthors{1} %  in this sample file, there are a *total*
\author{}
% \author[1]{Chris Kedzie}
% \author[1]{Kathleen McKeown}
% \author[2]{Fernando Diaz}
% \affil[1]{Columbia University, Department of Computer Science}
% \affil[2]{Microsoft Research}
%    Alternate {\ttlit ACM} SIG Proceedings Paper in LaTeX
%Format\titlenote{(Produces the permission block, and copyright information).
%For use with SIG-ALTERNATE.CLS. Supported by ACM.}} \subtitle{[Extended
%Abstract] \titlenote{A full version of this paper is available as
%\textit{Author's Guide to Preparing ACM SIG Proceedings Using
%\LaTeX$2_\epsilon$\ and BibTeX} at \texttt{www.acm.org/eaddress.htm}}}
%
% You need the command \numberofauthors to handle the 'placement and
% alignment' of the authors beneath the title.
%
% For aesthetic reasons, we recommend 'three authors at a time' i.e. three
% 'name/affiliation blocks' be placed beneath the title.
%
% NOTE: You are NOT restricted in how many 'rows' of "name/affiliations" may
% appear. We just ask that you restrict the number of 'columns' to three.
%
% Because of the available 'opening page real-estate' we ask you to refrain
% from putting more than six authors (two rows with three columns) beneath the
% article title.  More than six makes the first-page appear very cluttered
% indeed.
%
% Use the \alignauthor commands to handle the names and affiliations for an
% 'aesthetic maximum' of six authors.  Add names, affiliations, addresses for
% the seventh etc. author(s) as the argument for the \additionalauthors
% command.  These 'additional authors' will be output/set for you without
% further effort on your part as the last section in the body of your article
% BEFORE References or any Appendices.

%\numberofauthors{3} %  in this sample file, there are a *total*
% of EIGHT authors. SIX appear on the 'first-page' (for formatting reasons)
% and the remaining two appear in the \additionalauthors section.
%
%\author{
% You can go ahead and credit any number of authors here, e.g. one 'row of
% three' or two rows (consisting of one row of three and a second row of one,
% two or three).
%
% The command \alignauthor (no curly braces needed) should precede each author
% name, affiliation/snail-mail address and e-mail address. Additionally, tag
% each line of affiliation/address with \affaddr, and tag the e-mail address
% with \email.
%
% 1st. author

%\alignauthor Chris Kedzie\\ \affaddr{Columbia University}\\
%\affaddr{Department of Computer Science} \email{kedzie@cs.columbia.edu}

% 2nd. author
%\alignauthor Kathleen McKeown\\ \affaddr{Columbia University}\\
%\affaddr{Department of Computer Science}\\ \email{kathy@cs.columbia.edu}

% 3rd. author
%\alignauthor Fernando Diaz\\ \affaddr{Microsoft Research}\\
%\email{fdiaz@microsoft.com}

% There's nothing stopping you putting the seventh, eighth, etc.  author on
% the opening page (as the 'third row') but we ask, for aesthetic reasons that
% you place these 'additional authors' in the \additional authors block, viz.
%} 
\date{29 October 2014}
% Just remember to make sure that the TOTAL number of authors is the number
% that will appear on the first page PLUS the number that will appear in the
% \additionalauthors section.

\maketitle \begin{abstract} 
During crises such as natural disasters or other human tragedies, information needs of both civilians and responders often require urgent, specialized treatment.  Slow or ineffective information access can significantly impact the safety and health of individuals.  Monitoring and summarizing a text stream during such an event remains a difficult problem.  The temporal summarization task refers to extracting short, sentence-length updates about a seed event from a stream of text data.  We present a system which makes substantial improvements over the state of the art in temporal summarization.  Our approach predicts the salience of sentences with respect to an event and then uses these predictions to directly bias a clustering algorithm for sentence selection, increasing the novelty of the updates. We use novel, disaster-specific features for salience prediction, including geo-locations and language models representing the language typically used to describe different disaster types. 
We evaluate our system on a standard set of retrospective events using a combination of human judgements and automatic evaluation measures. We demonstrate the effect of different feature groups, and the importance of salience prediction for update precision.


\end{abstract}

% A category with the (minimum) three required fields
%\category{H.4}{Information Systems Applications}{Miscellaneous}
%A category including the fourth, optional field follows...
%\category{D.2.8}{Software Engineering}{Metrics}[complexity measures,
%performance measures]

%\terms{Summarization}

%\keywords{Extractive Summarization, Affinity Propagation}

\fdcomment{We should try to pull out as much of the TREC-specific language as possible (e.g. `run submissions', `participation in the TS track') to make the results more general.}
\label{sec:introduction}

During crises, information is critical for first responders, crisis management
organizations, and those caught in the event. When the event is significant, 
as in the case of Hurricane Sandy, the amount of content produced by 
traditional news outlets, government agencies, relief organizations, and 
social media can vastly overwhelm those trying to monitor the situation. 
Crisis informatics \cite{palen2010vision} is dedicated to finding methods for 
sharing the right information in a timely fashion during such an event.
Research in this field has focused on human-in-the-loop approaches ranging 
from on the ground information gathering to crowdsourced reporting and 
disaster management \cite{starbird2013working}.


Multi-document summarization has the potential to assist the crisis 
informatics community. Automatic summarization could deliver relevant and 
salient information at regular intervals, even when human volunteers are 
unable to. Perhaps more importantly it could help filter out unnecessary and 
irrelevant detail when the volume of incoming information is large. While 
methods for identifying, tracking, and summarizing events from text based 
input have been explored extensively
\cite{allan1998topic,Filatova&Hatzivassiloglou.04a,Wang&al.11}, 
these experiments were not developed to handle streaming data from a
heterogeneous environment at web scale. These methods also rely heavily on 
redundancy which is suboptimal for time sensitive domains where there is a 
high cost in delaying information.


In this paper, we present an update summarization system to track events
across time. Our system predicts sentence salience in the context of a
large-scale event, such as a disaster, and integrates these predictions into
a clustering based multi-document summarization system. We demonstrate that 
combining salience with clustering produces more relevant summaries compared 
to baselines using clustering or relevance alone.  Our experiments suggest 
that this is because our system is better able to adapt to dynamic changes in 
input volume that adversely affect methods that use redundancy as a proxy for 
salience. 


In addition to the tight integration between clustering and salience
prediction, our approach also exploits knowledge about the event to determine
salience. Thus, salience represents both how typical a sentence is of the  
event
type (e.g., industrial accident, hurricane, riot) and whether it specifies 
information
about this particular event. 
Our feature representation includes a set of language models, one for each
event type, to measure the typicality of the sentence with regard to the 
current event, the distance of mentioned locations from the center of
the event, and the change in word frequencies over the time of the event.
While we evaluate these features in the domain of disasters, this approach is 
generally applicable to many update summarization tasks.


Our approach achieves a statistically significant improvement in ROUGE scores 
compared to multiple baselines. Additionally, we introduce novel methods for 
estimating the average information gain each update provides and how 
completely the update summary covers the event it is tracking; our system's 
updates contain more relevant information on average than the competing 
baselines.


The remainder of the paper is organized as follows. We begin with a review of 
related work in the information retrieval and multi-document summarization 
literature. Section~\ref{sec:methods} outlines the details of our salience 
and summarization models. Next we describe our data (Section~\ref{sec:data}) 
and experiments (Section~\ref{sec:exper}). Finally, we discuss our results 
(Section~\ref{sec:results}) and conclude the paper.


\section{Related Work}
\label{sec:relatedwork}
A principal concern in extractive multi-document summarization is the
selection of salient sentences for inclusion in summary output
\cite{nenkova2012survey}.  This has often been approached as a ranking
problem.
%We broadly conceptualize this decision as either an intrinsic or extrinsic
%sentence evaluation process. Intrinsic approaches evaluate sentences
%individually, possibly by predicting the impact on summary quality using
%sentece level features. 
Sentences have been ranked by the average word probability, average tf-idf
score, and the number of topically related words (topic-signatures in the
summarization literature)
\cite{nenkova2005impact,hovy1998automated,lin2000automated}. The first two
statistics are easily computable from the input sentences, while the third
only requires an additional, generic background corpus.  Another ranking
approach, centroid summarization, involves creating an average bag of words
(BOW) vector, the centroid, from the input sentences and ranking sentences by
their similarity to the centroid \cite{radev2004centroid}.  Graph
\cite{erkan2004lexrank} and clustering
\cite{hatzivassiloglou2001simfinder,mckeown1999towards,siddharthan2004syntactic}
based approaches, on the other hand, make use of pair-wise similarity
comparisons amongst input sentences.  In these models, salient sentences are
more central to the input or cluster, respectively.

%identify salient regions of the input space while simultaneously coping with
%redundancy.  Graph-based algorithms have been used to rank sentences
%Clustering algorithms, e.g., are commonly used to exploit redundancy in
%input. Input sentences are clustered and summaries are generated by selecting
%the most representative sentence from each cluster.  Graph-based models have
%also been used for summarization.  E.g., the LexRank algorithm treats
%sentences as nodes in a graph, where edges are constructed by way of cosine
%similarity between sentence nodes; edges are either continuosly weighted by
%similarity or discrete, existing only when the similarity is above a
%threshold.  The PageRank algorithm is used on the graph to find the most
%important sentence nodes. In both clustering and graph-based approaches,
%sentence salience is largely determined by the pairwise relations between
%sentences.

Supervised learning has also been applied to this task. Model features are
usually derived from human generated summaries, and are non-lexical in nature
(e.g., sentence starting position, number of topic-signatures, number of
unique words, word frequencies). Seminal work in this area has employed naive
Bayes and logistic regression classifiers to identify sentences for summary
inclusion \cite{kupiec1995trainable,conroy2001using}. 

%\fdadd{
Several researchers have recognized the importance of summarization during
natural disasters.  Guo \textit{et al.} developed a system for detecting
novel, relevant, and comprehensive sentences immediately after a natural
disaster \cite{qi:temporal-summarization}.  The method uses a model of
sentence relevance and novelty in order to select appropriate updates.
Training data for regression targets is automatically generated from
retrospective Wikipedia data.  The system is evaluated on news documents
related to 197 natural and human disasters from 2009 to 2011 using variants of
Rouge modified to capture novelty, relevance, and comprehensiveness
\cite{lin2004rouge}.  Wang and Li present a clustering-based approach to
efficiency detect important updates during natural disasters
\cite{wang:update-summarization}.  The algorithm works by hierarchically
clustering sentences online, allowing the system to output a more expressive
narrative structure than Guo \textit{et al.}.  The method is evaluated on
official press releases related to Hurricane Wilma  in 2005 using Rouge score
between the system summary and a manually generated target summary.

\fdcomment{Glasgow temporal summarization system \cite{mccreadie:temporal-summarization}.}

\section{Motivation}

Most multi-document summarization approaches involve either sentence 
clustering or ranking or both to select summary sentences.
When the summarization task
is broad or underspecified, clustering is most appropriate in that the data
``speaks for itself.'' This is probably most true when summarization is 
interpreted to mean exploratory data analysis. 
When there are more constraints on the output, as in many text summarization
tasks,
it is often easier to design appropriate features for the ranking approach, 
either by rule or learning-to-rank. The primary contribution of this paper
is a framework for combining both of these approaches using affinity
propagation clustering. Additionally, we provide a task specific evaluation
of this clustering model on the TREC Temporal Summarization task. 


\ckcomment{Not sure how to present this together, we also essentially evaluate
a salience model by feature ablation for the TS task, but this feels very specific compared to the general framework idea.}

\fdcomment{I guess I think we need to two things in this section (and/or introduction).  1. explain why the problem matters.  2. explain why our approach is compelling.  }

\fdcomment{if easy to generate, would be good to have a single figure for an event that includes 1. the query, 2. the distribution of (filtered) document volume over time, and 3. a sample of the gold nuggets.  this provides the reader with a visual explanation of the problem, the solution, and the difficulty}

\label{sec:problemdefinition}
In order to evaluate an update summarization system, we adopt the simulation-based approach used in the TREC Temporal Summarization Track.  We provide a brief overview of the problem.  Details on the formulation can be found in the track overview paper \cite{aslam2013trec}.  

An update summarization system takes as input 
\begin{enumerate*}[label=\itshape\alph*\upshape)]
  \item a short query $\query$ defining the event to be tracked (e.g. `Hurricane Sandy'), 
  \item an event category $\eventcategory$ defining the type of event to be tracked (e.g. `hurricane'), 
  \item a stream of time-stamped documents, $(\doc_0, \doc_1,\ldots,\doc_T)$,
  presented in temporal order, and \item an evaluation time period of interest,
      $(\stime,\etime)$.  \end{enumerate*} While processing documents
      throughout the time period of interest, the system must output sentences,
      known as updates, which are \emph{relevant} to the query,
      \emph{comprehensive} with respect to the event, \emph{novel} to the user,
      and \emph{timely} with respect to when the update occurred.  Precisely
      how to measure these properties will be discussed in Section
      \ref{sec:methods}.  In our version of this problem, we assume that that
      the system receives one batch of new sentence-segmented documents every
      hour throughout the period of interest.

We address disaster types such as terrorism and mass shootings (e.g., the 2012
shooting in Aurora, Col.), natural disasters (e.g., Hurricane Sandy), accidents
(e.g., the 2012 Pakistan garment factory fire), and social activism (e.g., the
Arab spring). The size and scope of the events varies considerably. Hurricane
Sandy, for example, effected multiple countries over weeks while the
2012 Aurora shooting was contained to a single location spanning several hours.

%\kmcomment{This was as best as I could do on this. These are the example
%    categories from the wikipedia page. However you refer in the language model
%    description to the event type earthquake which is more specific. Also, in
%    the TREC data you refer to event types that are specified in the metadata.
%So I'm wondering how many different categories you have and where they come
%from.}

%\fdcomment{include description of nuggets here.} 
%%\kmcomment{We should jointly
%    discuss what goes in data and what goes in problem definition. I have moved
%    event types here because I think knowing how many types and examples of
%    them would be helpful upfront. I think where nuggets and wikipedia pages go
%is questionable.}

Figure \ref{alg:temporal-summarization} outlines our general update
summarization algorithm.  At each hour, the system processes each input 
sentence batch $S_t$.
We predict the salience $P$ for all input sentences $s\in S_t$.
Next, we cluster $S_t$ using the AP clustering algorithm, biased by $P$,
obtaining a set of exemeplar sentences $E$. Finally we select a subset of $E$
to be updates and add those to our set of summary updates $U$.

\begin{algorithm}%[H]
 \KwData{ 
     $S_{\stime}, S_{\stime+1},\ldots,S_{\etime} $ --- time ordered sentence
     batches\\
 \KwResult{U --- a list of updates, i.e. the update summary} 
}
 ~\\
 Initialize empty list $U$ of updates\\
    \For{$t \gets \stime,\ldots,\etime $}{
        $P \gets \operatorname{PredictSalience}(S_t)$\\
        $E \gets \operatorname{APCluster}(P, S_t)$ \\
        $U_t \gets \operatorname{SelectUpdates}(E,P,U)$\\
        $U \gets U \cup U_t$
}
 \caption{Temporal Summarization Algorithm}\label{alg:temporal-summarization}
\end{algorithm}

\subsection{Salience Prediction}
%
\begin{figure}[t!]
\begin{tabular}{| l |} 
\hline
\textbf{Basic Features}\\
$\cdot$ Sent. position (normalized by doc. length) \\
$\cdot$ Sent. length \\
$\cdot$ Ratio of punc. to non-punc. chars. \\
$\cdot$ Ratio of caps. to non-caps. chars. \\
$\cdot$ Ratio of lowercase to other chars. \\
$\cdot$ Ratio of uppercase to other chars. \\
$\cdot$ \# of caps. words (normalized by \# of words)\\
$\cdot$ \# of Person, Location, Org. Date, Number,\\
$\;\;$ Ordinal, Percent, Money, Set, Misc  N.E. tags \\
$\;\;$ (normalized by \# of words)\\
\hline
\textbf{Query Features}\\
$\cdot$ \% of query words covered by sent.\\
$\cdot$ Total query matches.\\
$\cdot$ Total event-type synonyms/hypernyms/hyponyms\\
$\;\;$ coverage.\\
$\cdot$ Total event-type synonyms/hypernyms/hyponyms\\
$\;\;$ matches.\\
\hline
\textbf{Language Model Features}\\
$\cdot$ Avg. token log probability (domain lang. model)\\
$\cdot$ Avg. token log probability (background lang. model)\\
\hline
\textbf{Geo-tag Features}\\
$\cdot$ Median document distance to nearest location \\
$\;\;$ cluster (current hour).\\
$\cdot$ Median document distance to nearest location \\
$\;\;$ clusters (previous hour).\\
$\cdot$ Distance of first location in doc. to nearest location \\
$\;\;$ cluster (current hour).\\
$\cdot$ Distance of first location in doc. to nearest location \\
$\;\;$ cluster (previous hour).\\
\hline
\textbf{Temporal Features}\\
$\cdot$ Avg. tf-idf at current time.\\
$\cdot$ Change in avg. tf-idf since previous hour (up to\\
$\;\;$  24 hours).\\
$\cdot$ Time since query/event start.\\
\hline
\end{tabular}
\caption{Salience model features.}
\end{figure}

\subsubsection{Features}
We want our model to be predictive across different kinds of events so we avoid lexical features.  Instead, we extract a variety of features including language model scores, geographic relevance, and temporal relevance from each sentence.  These features are used to fit a Gaussian process regression model that can predict the similarity of a sentence to a gold summary \cite{preotiuc2013temporal}.  
\fdcomment{can we enumerate all of these?}

\paragraph{Basic Features}
%KM - Would be good to have quick justification of these features. I added a
%sentence. Feel free to edit or remove.

We employ several basic features that have been used previously in supervised models to rank sentence salience \cite{kupiec1995trainable,conroy2001using}. These include sentence length, the number of capitalized words normalized by sentence length, document position, number of named entities.  
Since training is one on grammatical English, some of these features help
to downweight sentences that are ungrammatical (e.g., have too many capitalized words or are too short).
Others help to more heavily weight important sentences (e.g., that appear in
prominent positions such as paragraph initial or article initial).

\paragraph{Query Features}

Query features measure the relationship between the sentence and the event query and type.  These include the number of query words present in the sentence in addition to the number of event type synonyms, hypernyms, and hyponyms as found in WordNet \cite{miller1995wordnet}.  For example, for event type \emph{earthquake},  we match sentence terms ``quake'', ``temblor'', ``seism'', and ``aftershock''.
\paragraph{Language Model Features}\label{subsubsec:lm}
Language models allow us to measure the likelihood of a sentence having been produced from a particular source.  We consider two types of language model features.  The first model is estimated from a corpus of generic news articles.  This model is intended to assess the general writing quality (grammaticality, word usage) of an input sentence and helps us to filter out text snippets which are not sentences (e.g., web page titles).  The second model is estimated from text specific to our event types.  For example, the language model for event type `earthquake' is estimated from Wikipedia pages under the category \emph{Category:Earthquakes}.  These models are intended to detect sentences similar to those appearing in summaries of other events in the same category (e.g. most earthquake summaries are likely to include higher probability for ngrams including the token `magnitude').  
%KM - Note: Someplace the exact list of event types should appear. Probably not
%here.
%KM - I note you have it in a later section but it is labeled as data you use
%to train language models and semantic similarity. I think it would be good to
%have up front in definition of task.


%For both models, we Finally, we extract the percentage of capitalized words,
%and sentence length as features. These last two features also help to
%identify sentences that are less likely to contain relevant content-- overly
%long and heavily capitalized sentences in our corpus were likely to be long
%strings of web-page headlines, section headers, and other irrelevant page
%structure. 

\paragraph{Geographic Relevance Features}

Locations are identified using a named entity tagger. For each location in a sentence, we obtain its latitude and longitude using the a publicly available geolocation service.  We then compute its distance to that of the event location.  It is possible for a sentence and an event to have multiple locations so we take as features the minimum, maximum, and average distance of all sentence-event location pairs.  Distances are calculated using the Vincenty distance. 
%KM - Probably should say how you determine event location. Some events move.
%KM - In your figure you only include one distance.

\paragraph{Temporal Relevance Features}

Our data consists of hourly crawls of online content and so we exploit the temporality of corpus by capturing the burstiness of a sentence, i.e.  the change in word frequency from one hour to the next.``Bursty'' sentences often indicate new and important data. 

Let $D_t$ be the set of web pages at time $t$ and let $s = \{w_1,\ldots,w_n\}$ be a sentence from a page $d \in D_t$.  We calculate the 1-hour burstiness of sentence $s$ from document $d$ at hour $t$  as 
\begin{align*}
\operatorname{b}_1(s,d,t) = \frac{1}{|s|} \sum_{w \in s} \Bigg( &
\operatorname{tf-idf}_t(w,d)  \\ & \left. - \frac{\sum_{d^\prime \in D_{t-1}:
w \in d^\prime } \operatorname{tf-idf}_{t-1}(w,d^\prime)}{|\{d^\prime \in
D_{t-1}: w \in d^\prime\}|} \right) \end{align*}

where \begin{align*} \operatorname{tf-idf}_t(w,d) =&
\log\left(1+\sum_{w^\prime \in d}1\{w=w^\prime\}  \right)\\ & \times
\log\left(\frac{|D_t|}{1 + \sum_{d^\prime \in D_t}1\{w \in d^\prime\}}\right).
\end{align*}
% 1\{w = w^\prime} %- \operatorname{avg-tf-idf}_{t_{i-1}}(w).
%\end{align*}


We similarly find the sentence's 5-hour burstiness.  In addition to burstiness, we also include the sentence's average tf-idf and hours since the event in question started as features.

\subsubsection{Model}
%KM - I think one of the words ``sentence'' below should be something else, btu
%not sure what you meant. Ah.. I think ``salience''?
%We adopt Gaussian process regression in order to predict sentence sentence
We adopt Gaussian process regression in order to predict sentence salience
\cite{rasmussen:gaussian-process-book}.  Gaussian process regressors are a
class of data-driven, non-parametric model generalizing the multi-variate
Gaussian to the infinite dimensional setting.  Gaussian processes are general
and are state of the art for many regression tasks.  A full treatment of
Gaussian process regression is beyond the scope of this article and can be
found in standard textbooks.


Being non-parametric, the Gaussian process relies on a  covariance matrix $\kernelMatrix$, measuring the affinity between pairs of instances, in our case candidate sentences.  In our experiments, we used a radial basis function (RBF) kernel.  Given two featurized sentences, $\features$ and $\features^\prime$, the RBF kernel is defined as,
\begin{align*}
        \kernelMatrixij{\features}{\features^\prime}&= \sigma^2 \exp\left(- \frac{1}{2} 
\sum_{i=1}^{\numfeatures} \frac{ (\featuresi{i}-\featuresi{i}^\prime)^2}{\ell_i^2} \right)
\end{align*}
where $\sigma$ and the $\ell_i$ are parameters we fit to our observed training data. The $\ell_i$ are feature dependent scaling parameters; once learned, they not only improve the accuracy of the model, but give us some introspection  into which features are more important.


% Formally, let $p(f)$ be a distribution over functions where $f$ is any mapping
% of an input space $\mathcal{X}$ to the reals,
%
% $$f: \mathcal{X} \rightarrow \mathcal{R}.$$
% Let the random variable $\mathbf{f} = (f(x_1),\ldots,f(x_n) )$ be
%  an $n$-dimensional vector whose elements are evaluations of the function $f$
% at points $x_i \in \mathcal{X}$.
% We say $p(f)$ is a Gaussian process if for any finite subset
% $\{x_1,\ldots,x_n\} \subset \mathcal{X}$, the marginal distribution over
% that finite subset $p(\mathbf{f})$ has a multivariate Gaussian distribution.
% A GP is parameterized by a mean function $\mu(\mathbf{x})$ and a
% covariance function $K(x,x^\prime)$. Generally, the mean function is simply
% set to 0, leaving the distribution to be completely characterized by the
% kernel function on the data.
%
% In the regression setting, we typically have a response variable $y$ that
% is the sum of our model prediction  and
% some Gaussian noise, i.e. $y = f(x) + \epsilon$ with
% $\epsilon \sim \mathcal{N}(0, \sigma^2)$. When
% $f \sim \operatorname{GP}(\mathbf{0}, \mathbf{K})$, the
% two distributions
% of principal interest are the marginal likelihood
% $p(\mathbf{y}|\mathbf{X}) =
% \mathcal{N}(\mathbf{0},\mathbf{K} + \sigma^2\mathbf{I})$ and the predictive
% distribution,
%
% $$p(\mathbf{y_*}|\mathbf{x_*},\mathbf{X},\mathbf{y}) =
% \mathcal{N}(\boldsymbol{\mu}_*, \boldsymbol{\sigma}^2_*) $$
%
% where $\mathbf{x_*}$ is a new or unseen input, $\mathbf{y_*}$ our predicted
% response, and
% \begin{align*}
% \boldsymbol{\mu}_* & = \mathbf{K_*}(\mathbf{K} + \sigma^2\mathbf{I})^{-1}\mathbf{y} \\
% \boldsymbol{\sigma}^2_* &
% = \mathbf{K}_{**} - \mathbf{K}_*(\mathbf{K} + \sigma^2\mathbf{I})^{-1}
% \mathbf{K}_*^T + \sigma^2\\
% \end{align*}.
%
% Here $\mathbf{K}_* = K(\mathbf{x}_*, \mathbf{X})$, and
% $\mathbf{K}_{**} = K(\mathbf{x}_*, \mathbf{x}_*)$.
%
%


\section{Algorithms}\label{sec:background}

\subsection{Affinity Propagation}

Affinity propagation (AP) is a message passing algorithm that identifies both
exemplar data points and assignments of each point to an exemplar.  This is
done iteratively by passing \emph{responsibility} and \emph{availability}
messages between data points that quantify the fitness of one data point to
represent another, and the fitness of a data point to be represented based on
the choices of other data points respectively \cite{dueck2007non}.

AP is parameterized by an $n\times n$ similarity matrix $S$ and an $n\times 1$
preference vector $\pi$.  $S$ is a real-valued matrix where $S(i,j)$ is the
similarity of the $i$-th data point to the $j$-th data point.  $S$ does not
need to be symmetric.  $\pi$ is a real-valued vector where $\pi(i)$ expresses
our preference that the $i$-th data point can serve as an exemplar a priori of
other data points. 




AP has several useful properties that comport well with the TS track 
requirements. First, the number of clusters $k$ is not a hyper-parameter
of the model. Since we will be running this algorithm many times in per run,
the usual methods of hyperparameter search become infeasible 
(?). The number of clusters falls out of the algorithm
organically-- lower overall preference values will result in fewer clusters. 

Secondly, the arbitrary nature of the preferences
allow us to incorporate a variety of signals for
identifying the best exemplars, i.e. salience and redundancy signals. 
The preferences can be thought of as self-similarities (similarities that
pairwise-comparison based algorithms would ignore) that we can exploit to 
incorporate our prior beliefs about a data point.

Finally, cluster exemplars are guaranteed to be actual data points. Many 
clustering algorithms group data around mathematical objects (e.g., the
mean) that are not necessarily observed in the data. The extractive nature of
this TS task requires that we emit actual data points (i.e. sentences).
We are able to sidestep 
the additional requirement of selecting a most
representative cluster member, as this is computed explicitly in the AP 
algorithm. 




%Each element $R(i,k)$ expresses the fitness of the $k^{th}$ point to serve as
%the exemplar of the $i^{th}$ point relative to other potential exemplars.
%Each element $A(i, k)$ represents the $k^{th}$ point's ``availability'' to
%serve as an exemplar of the $i^{th}$ point, taking into account other points'
%preference for the $k^{th}$ point as an exemplar.



%While, AP does not set the number of exemplars before-hand, a lower overall
%preference values will result in a smaller number of exemplars.


%arbitrary pair-wise similarity function $S: \mathcal{R}^d \rightarrow
%\mathcal{R}$, where $d$ is the dimension of the data being clustered and a
%real-valued preference $\pi_i$ quantifying our belief a priori of the
%$i^{th}$ element's ability to serve as an exemplar.




\subsection{Semantic Similarity}\label{subsec:semsim}

Whenever we make a pairwise comparison between sentences, we use the weighted
textual matrix factorization (WTMF) model of \cite{guo2012simple}. This 
model can be thought of as a variant of latent semantic analysis (?), 
where words that are not present in a sentence are explicitly modeled.
More formally, we have a term-sentence matrix 
$\mathbf{X}\in\mathcal{R}^{v \times n}$ representing $n$ sentences with a 
vocabulary of $v$ words; $\mathbf{X}_{i,j}$ indicates is non-zero if sentence
$j$ contains word $i$. In the WTMF regime, we want to find an approximation
of $\mathbf{X} \approx \mathbf{P}^T\mathbf{Q}$, where 
$\mathbf{P} \in \mathcal{R}^{k \times v}$ is a latent word vector space and
$\mathbf{Q} \in \mathcal{R}^{k \times n}$ is a latent sentence vector
space. These matrices are found by minimizing the objective function

$$\sum_i^v \sum_j^n \mathbf{W}_{i,j}(\mathbf{P}_{\cdot,i}^T
\mathbf{Q}_{\cdot,j} 
- \mathbf{X}_{i,j})^2 
 + \lambda ||\mathbf{P}||_2^2 + \lambda ||\mathbf{Q}||_2^2$$

where $\mathbf{W}_{i,j} = 
\begin{cases} 1, & \textrm{if $\mathbf{X}_{i,j} \ne 0$ } \\
w_m, & \textrm{if $\mathbf{X}_{i,j} = 0$ }\\
\end{cases}$
and $\lambda$ is a hyperparameter controlling the regularization terms.

The $w_m$ term is another model hyperparameter that is set to a small constant
($\le .01$). The weight matrix $\mathbf{W}$ has the effect of discounting the
reconstruction error of missing terms (words that did not occur a sentence).

Given an unseen sentence $\hat{i}$ we can project its term vector into the
latent sentence vector space with 

$$
\mathbf{Q}_{\cdot,\hat{i}} = (\mathbf{P}\mathbf{\tilde{W}}^{(\hat{i})}
\mathbf{P}^T  + \lambda\mathbf{I} )^{-1} 
\mathbf{P}\mathbf{\tilde{W}}^{(\hat{i})} \mathbf{X}_{\cdot, \hat{i}}
$$  

where $\mathbf{\tilde{W}}^{(\hat{i})}$ is an $v\times v$ diagonal matrix
where $\mathbf{\tilde{W}}^{(\hat{i})}_{j,j}$ is equal to $1$ or $w_m$ 
depending on whether or not the $j$-th term occurs in sentence $\hat{i}$.


The WTMF model is used extensively throughout our TS system. When making any 
pairwise comparison between sentence $i$ and $j$, we first construct
their latent sentence vectors $\mathbf{Q}_{\cdot,i}$ and
$\mathbf{Q}_{\cdot,j}$ and then find the cosine similarity 
$\displaystyle \operatorname{cos-sim}
(\mathbf{Q}_{\cdot,i}, \mathbf{Q}_{\cdot,j}) = 
\frac{\mathbf{Q}_{\cdot,i}^T\mathbf{Q}_{\cdot,j}}{||\mathbf{Q}_{\cdot,i}||_2
||\mathbf{Q}_{\cdot,j}||_2   }$.


Because the events for the TS task come from different domains, we construct
domain specific latent word vector spaces for each domain using in-domain 
Wikipedia pages (see~\cref{sec:data} for more details).

\subsection{Gaussian Processes}

A Gaussian process (GP) is a distribution over functions and is a 
generalization of the multi-variate Gaussian to the infinite dimensional
setting. That is, we use the observed data to define a distribution over 
possible functions that generated this data, without having to explicitly 
parameterize the function---in this sense GPs are considered 
a non-parametric model.

Formally, let $p(f)$ be a distribution over functions where $f$ is any mapping
of an input space $\mathcal{X}$ to the reals,

$$f: \mathcal{X} \rightarrow \mathcal{R}.$$ 
Let the random variable $\mathbf{f} = (f(x_1),\ldots,f(x_n) )$ be
 an $n$-dimensional vector whose elements are evaluations of the function $f$
at points $x_i \in \mathcal{X}$.
We say $p(f)$ is a Gaussian process if for any finite subset 
$\{x_1,\ldots,x_n\} \subset \mathcal{X}$, the marginal distribution over 
that finite subset $p(\mathbf{f})$ has a multivariate Gaussian distribution.
A GP is parameterized by a mean function $\mu(\mathbf{x})$ and a 
covariance function $K(x,x^\prime)$. Generally, the mean function is simply
set to 0, leaving the distribution to be completely characterized by the
kernel function on the data.

In the regression setting, we typically have a response variable $y$ that
is the sum of our model prediction  and 
some Gaussian noise, i.e. $y = f(x) + \epsilon$ with 
$\epsilon \sim \mathcal{N}(0, \sigma^2)$. When
$f \sim \operatorname{GP}(\mathbf{0}, \mathbf{K})$, the
two distributions
of principal interest are the marginal likelihood
$p(\mathbf{y}|\mathbf{X}) = 
\mathcal{N}(\mathbf{0},\mathbf{K} + \sigma^2\mathbf{I})$ and the predictive
distribution,

$$p(\mathbf{y_*}|\mathbf{x_*},\mathbf{X},\mathbf{y}) =
\mathcal{N}(\boldsymbol{\mu}_*, \boldsymbol{\sigma}^2_*) $$

where $\mathbf{x_*}$ is a new or unseen input, $\mathbf{y_*}$ our predicted
response, and
\begin{align*}
\boldsymbol{\mu}_* & = \mathbf{K_*}(\mathbf{K} + \sigma^2\mathbf{I})^{-1}\mathbf{y} \\
\boldsymbol{\sigma}^2_* & 
= \mathbf{K}_{**} - \mathbf{K}_*(\mathbf{K} + \sigma^2\mathbf{I})^{-1}
\mathbf{K}_*^T + \sigma^2\\
\end{align*}.

Here $\mathbf{K}_* = K(\mathbf{x}_*, \mathbf{X})$, and 
$\mathbf{K}_{**} = K(\mathbf{x}_*, \mathbf{x}_*)$.


GP's are incredibly general, and are state of the art for many regression 
tasks (?). The reliance on the covariance matrix 
$\mathbf{K}$ for parameterization opens up the wide world of kernel methods
for regression, and many varieties of similarity functions can be used.
In our experiments we used a radial basis function kernel 
$$K(\mathbf{x},\mathbf{x}^\prime) = \sigma^2 \exp\bigg(- \frac{1}{2} 
\sum_{i=1}^d \frac{ (x_i-x^\prime_i)^2}{\ell_i^2} \bigg)$$ where 
$\sigma$ and the $\ell_i$ are parameters we fit to our observed training data.
The $\ell_i$ are feature dependent scaling parameters; once learned, they not
only improve the accuracy of the model, but give us some introspection 
into which features are more important.

%the formation of sentence clusters around more salient sentences. We
%introduce the affinity propagation algorithm as as an elegant way to
%incorporate our salience predictions into a

%What follows is a description of our ongoing event summarization efforts.  We
%briefly situate our approach to summarization within the broader field of
%multi-document summarization, and then introduce the affinity propagation
%algorithm which we use for clustering. This algorithm allows us to elegantly
%address the salient sentence selection problem by incorporating our prior
%beliefs about sentence quality. Next, we describe out method for modeling
%summary sentence quality, and the features used in this model.  Finally, we
%address future features and system improvements that we are incorporating
%into our summarizer.



\subsection{Temporal Summarization System}\label{sec:approach}


We first begin with some notation. For a given event, let $\corpus$ be the set
of retrieved documents. A document $\doc \in \corpus$ is an ordered sequence
of sentences $\{\sent_{1,\doc},\ldots,\sent_{|\doc|,\doc} \}$. 
Additionally, each document has a timestamp $\dtime(\doc)$. Finally, let 
$\corpus_{\hour_i}$ be the set of retrieved documents such that 
$\hour_i \le \dtime(\doc) < \hour_{i+1}$ for all $\doc \in \corpus_{\hour_i}$.

%Our TS system involves ? phases involving sentence level classification, 
%regression, and clustering. 
Figure ? outlines our general temporal summarization algorithm. The description
of our approach is as follows.
For each event, we iterate over the retrieved 
documents in hourly chunks, emitting 0 or more updates at each hour.
At each hour $\hour$, we process each document
$\doc \in \corpus_{\hour}$. First, we identify where the document content
actually is; Second, we predict the salience of all content sentences.
To account for redundancy, the predicted salience is penalized based on the 
distance of the sentence in question to the previous summary updates.

Finally, we cluster all content sentences 
for the current hour using the penalized salience predictions to bias the 
formation
of clusters around the most salient sentences. For each cluster center, or 
exemplar, that results, we check that the salience is above a threshold and 
that it does not belong to a singleton cluster; exemplars that satisfy these
conditions are emitted as an update.
Additionally, we maintain the complete set of updates in order to penalize
salience predictions in the subsequent time steps. 

% \begin{figure}
% \centering
\begin{algorithm}%[H]
 \KwData{$Query$ --- the set of event query words\\ 
        $SC$ --- the stream corpus\\

         
}
 ~\\
 Initialize empty list $\mathbf{U}$ of updates\\
 Initialize empty list $\boldsymbol{\Pref}^{(U)}$ of update preferences\\
 $\corpus \gets $ RetrieveDocuments($Query$, $SC$) (See \cref{subsec:Document Retrieval})\\
 \For{$i \gets 1,\ldots,t $}{

  Initialize empty lists $\SMat, \Pref$ \;
      

  \For{$\doc \in \corpus_{\hour_i}$}{
   \For{$\sent \in \operatorname{getContent}(\doc, Query)$ (See \cref{subsec:Content Detection})\\}{
     $\SMat.\operatorname{append}(\sent)$\;
     $\sigma \gets \operatorname{PredictSalience}(\sent)$ (See \cref{subsec:Predict})\\     
     $\Pref.\operatorname{append}(\sigma)$\;
     
   }        
  }
  %$\Sim \gets \operatorname{ComputeSimilarityMatrix}(X)$\;
  %$\operatorname{}$
  ~\\ 
  $\mathbf{U}_{h_i}, \Pref^{(U)}_{h_i} \gets \operatorname{SentenceSelection}
    (\SMat, \Pref)$ (See \cref{subsec:SentenceSelection})\\
  ~\\
  $\operatorname{Emit}(\mathbf{U}_{h_i})$ \\ 
  $\mathbf{U}\operatorname{.append}(\mathbf{U}_{h_i})$ \\
  $\mathbf{\Pref}^{(U)}\operatorname{.append}(\Pref^{(U)}_{h_i})$ \\
%gets \Updates_{cache} \cup \Updates_{\hour}$\;

 } 
 \caption{Temporal Summarization Algorithm}
\end{algorithm}
% \end{figure}

\subsubsection{Document Retrieval}\label{subsec:Document Retrieval}

The focus of our system was on salience prediction and clustering stages, and
so we relied heavily on the pre-filtered corpus provided by the track 
organizers. The TREC Temporal Summarization 2014 (TREC-TS-2014F) corpus is
a subset of the full TREC 2014 StreamCorpus and is
intended to be a high recall retrieval of documents related to all 15 of 
the 2014 TS events. 

Track participants were provided with a set of query words for each event.
For example, the event ``Costa Concordia disaster and recovery'' had query
words [``costa'', ``concordia''].
In order to construct an event specific corpus,
we retrieve all documents whose timestamps fall within the event start/stop 
times and whose raw html content contains at least one keyword from the 
event's query words. We further restricted our document set to only those 
articles from the news domain.


\subsubsection{Content Detection}\label{subsec:Content Detection}

In early versions of our summarization system, we found that structural html
artifacts and sentence tokenization errors
were negatively effecting the performance of later stages. Examples of the 
former include strings of link text like ``World Politics Sports ...'', while
examples of the latter included concatenations of various article headlines,
i.e. headlines pertaining to the event in question, as well as headlines from
other non-related events.
Both types of ``sentences'' were problematic for computing sentence similarity
as they were more likely to have higher average similarity to all input 
sentences. In turn they would be more likely to appear as cluster exemplars
in our clustering stage. From the clustering algorithm's point of view, this
is the correct decision to make---such multi-topic sentences are more general 
than a single topic, and better able to represent all aspects contained in a 
cluster. They make poor choices as updates, however, as they contain 
irrelevant information.

In order to filter out these problematic inputs, we trained a classifier to 
identify which sentences came from inside a document's main article and which
came from various headers, titles, menus, and links to other content. We collected ? random sentences and manually labeled whether the sentence came from
inside or outside the document's main article. We then trained a logistic 
regression classifier using the following features :

\begin{itemize}
 \item the position of the sentence
 \item word counts
 \item the last token in the sentence
 \item the last two tokens in the sentence
 \item the last three tokens in the sentence.
\end{itemize} 

These features were sufficient to capture 
the main difference between content and non-content sentences,
which 
was that content sentences generally ended with sentence final punctuation, 
i.e. periods or a closing quotation mark. 

Within our larger summarization framework, we process a document at a time,
identifying the subset of its sentences that are content sentences.
We then check to make sure \emph{all} event
query terms can be found within the document's content sentences. If so,
we send the content sentences on to the next stage of our pipeline; otherwise
we ignore all sentences in this document.

\subsubsection{Predicting Sentence Salience}\label{subsec:Predict}

In order to use AP clustering for summarization, we need to assign a
preference value to each input sentence.  In our approach, we equate a
sentence's salience with its preference.  A good model of sentence salience
should predict higher values for sentences that are more likely to appear in a
human generated summary of the event. 

We do not have such human judgments, but we do have last year's gold nugget
sentences.
The response variable we try to predict is a sentence's semantic similarity
(see ?) to the gold nugget sentences, i.e. we want to predict the similarity
to a gold nugget when the gold nugget is not known.


To build training data for this regression task, 
we use the TREC 2013 event/gold nuggets. For each event, we retrieve all 
sentences using the document retrieval and content selection steps outlined
in the previous selection. We then sample with replacement 200 sentences
from this collection, and extract non-lexical features (described in more 
detail below) for each sentence
to construct our design matrix $\mathbf{X}$. To build our response variables
$\mathbf{y}$ we compute the maximum semantic similarity of each input
sentence to the gold nugget sentences. We fit a $\operatorname{GP}$ with
an RBF kernel to this data, optimizing kernel parameters with the scaled 
conjugate gradient method. This sampling procedure is repeated 100 times
for each of the ? 2013 TS events, yielding ? total models. In our salience 
predictions for the 2014 events, we take the mean prediction of all models.

 
%we take a subset of sentences
%relevant to the TREC events (approximately 1000) and match them to the gold
%nugget sentence with highest similarity as determined by the sentence
%similarity system of \cite{guo2012simple}. 
%\cite{} have used this system previously to correlate sentences to meaningful
%units of information in human generated summaries. 
%We use the real-valued similarity scores as our salience scores for the
%training sentences.

We want our model to be predictive across different kinds of events so we
avoid lexical features.  Instead, we extract a variety of features including
language model scores, geographic relevance, and temporal relevance from each
sentence.  These features are used to fit a Gaussian process regression model
that can predict the similarity of a sentence to a gold summary
\cite{preotiuc2013temporal}.  
%We use the model predicted salience of each
%sentence as it's preference value in the AP clustering. 

\subsubsection{Basic Features}

We employ several basic features that have been used previously in supervised
models to rank sentence salience \cite{kupiec1995trainable,conroy2001using}.
These include sentence length, the number of capitalized words normalized by
sentence length, document position, number of named entities, etc.

\subsubsection{Query Features}

These features are derived from the event query and the event type.
These include the number of query words present in the sentence
in addition to the number of event type synonyms, hypernyms, and hyponyms
as found in WordNet \cite{miller1995wordnet}.
E.g., for event type \emph{earthquake}, e.g., we obtain ``quake,'' ``temblor,''
``seism,'' ``aftershock,'' etc.   


\subsubsection{Language Model Features}\label{subsubsec:lm}

%Because the data in our experiments is scraped from the web, it is common to
%find sentences that contain both salient informantion and two kinds of noise:
%noisey fragments of web page structure (e.g. section titles, \emph{News},
%\emph{Sports}, etc.) and references to other news not relevant to the topic
%summary.
%
We use two trigram language models, trained using the SRILM toolkit
\cite{stolcke2002srilm}, taking as features the average log probability (i.e.
the sentence's total log probability normalized by sentence length) from each
model.  This first model is trained on 4 years (2005-2009) of articles from
the Gigaword corpus.  Specifically, we use articles from the Associated Press
and the New York Times. This model is intended to assess the general writing
quality (grammaticality, word usage) of an input sentence and helps us to
filter out text snippets which are not sentences (e.g., web page titles).  The
second model is a domain specific language model. We build a corpus of
Wikipedia articles for each event type, consisting of documents from a related
Wikipedia category. E.g. for earthquakes, we collect pages under the category
\emph{Category:Earthquakes}. This model assigns higher probability to
sentences that are focused on the given domain.

%For both models, we Finally, we extract the percentage of capitalized words,
%and sentence length as features. These last two features also help to
%identify sentences that are less likely to contain relevant content-- overly
%long and heavily capitalized sentences in our corpus were likely to be long
%strings of web-page headlines, section headers, and other irrelevant page
%structure. 

\subsubsection{Geographic Relevance Features}

Locations are identified using a named entity tagger. For each location in a
sentence, we obtain its latitude and longitude using the Google Maps API.  We
then compute its distance to that of the event location.  It is possible for a
sentence and an event to have multiple locations so we take as features the
minimum, maximum, and average distance of all sentence-event location pairs.
Distances are calculated using the Vincenty distance.

\subsubsection{Temporal Relevance Features}

Our data consists of hourly crawls of online content and so we exploit the
temporality of corpus by capturing the burstiness of a sentence, i.e.  the
change in word frequency from one hour to the next.``Bursty'' sentences often
indicate new and important data. 

Let $D_t$ be the set of web pages at time $t$ and let $s = \{w_1,\ldots,
w_n\}$ be a sentence from a page $d \in D_t$.  We calculate the 1-hour
burstiness of sentence $s$ from document $d$ at hour $t$  as \begin{align*}
\operatorname{b}_1(s,d,t) = \frac{1}{|s|} \sum_{w \in s} \Bigg( &
\operatorname{tf-idf}_t(w,d)  \\ & \left. - \frac{\sum_{d^\prime \in D_{t-1}:
w \in d^\prime } \operatorname{tf-idf}_{t-1}(w,d^\prime)}{|\{d^\prime \in
D_{t-1}: w \in d^\prime\}|} \right) \end{align*}

where \begin{align*} \operatorname{tf-idf}_t(w,d) =&
\log\left(1+\sum_{w^\prime \in d}1\{w=w^\prime\}  \right)\\ & \times
\log\left(\frac{|D_t|}{1 + \sum_{d^\prime \in D_t}1\{w \in d^\prime\}}\right).
\end{align*}
% 1\{w = w^\prime} %- \operatorname{avg-tf-idf}_{t_{i-1}}(w).
%\end{align*}


We similarly find the sentence's 5-hour burstiness.  In addition to
burstiness, we also include the sentence's average tf-idf and hours since the
event in question started as features.


\subsubsection{Sentence Selection}\label{subsec:SentenceSelection}

In the sentence selection stage, we use the salience predictions from our GP
model as preferences in the AP clustering algorithm. The AP algorithm is 
parameterized by a similarity matrix $\mathbf{S}$ and a vector of 
preferences $\boldsymbol{\pi}$; we found AP to be very sensitive to these
parameters, and did not perform robustly on our range of inputs.
In order to improve the quality of the clusters and exemplar selection,
we re-scaled both the raw inputs $\mathbf{S}$ and $\boldsymbol{\pi}$. 
The raw preferences are scaled to lie within the  range $(-3, -2)$

The initial matrix $\mathbf{S}$ is computed by finding the pairwise semantic
similarity between input sentences. Self-similarities and similarities below 
a threshold $\lambda$ were masked and the remaining values scaled to the range
$(-3, -1)$.

\subsubsection{Preferences}

Before rescaling, we first penalize each $\boldsymbol{\pi}_i$ based on
the aggregate similarity of input sentence $\mathbf{s}_i$ to the set of previous updates
$\mathbf{U}$. We call this the redundancy penalty 

$$\rho_i = \sum_j \frac{\boldsymbol{\pi}^{(U)}_{j}\operatorname{cos-sim}(\mathbf{s}_i, \mathbf{U}_j)}
{\sum_{j^\prime}\operatorname{cos-sim}(\mathbf{s}_i, \mathbf{U}_{j^\prime})} $$   
where $\boldsymbol{\pi}^{(U)}_{j}$ is the salience prediction of the $j$ 
previous update. We calculate our new penalized preferences 
$\boldsymbol{\pi}^{(\rho)}$ where $\boldsymbol{\pi}^{(\rho)}_i = \boldsymbol{\pi}_i - \rho_i$.
Finally, we rescale $\boldsymbol{\pi}^{(\rho)}$ such that all values lie 
within the range $(-3,-2)$.

\subsubsection{Filtering}

We run AP clustering with our rescaled and penalized sentence similarities
and preferences. When the clustering has converged we emit all exemplar
sentences (cluster centers) whose preference is $\lambda$ standard deviations
above the mean preference for the current time period and who do not belong
to singleton clusters as updates. These updates and their preferences are 
retained for penalty calculations in subsequent time steps.




%In order to evaluate an update summarization system, we adopt the simulation-based approach used in the TREC Temporal Summarization (TS) Track.  We provide a brief overview of the problem.  Details on the formulation can be found in the track overview paper \cite{aslam2013trec}.  

Our update summarization system takes as input 
\begin{enumerate*}[label=\itshape\alph*\upshape)]
  \item a short query defining the event to be tracked (e.g. `Hurricane Sandy'), 
  \item an event category defining the type of event to be tracked (e.g. `hurricane'), 
  \item a stream of time-stamped documents %, $(\doc_0, \doc_1,\ldots,\doc_T)$,
  presented in temporal order, and \item an evaluation time period of interest.
    %  $(\stime,\etime)$.  
\end{enumerate*} 
While processing documents
throughout the time period of interest, the system outputs sentences
from these documents likely to be useful to query issuer.  We refer
to these selected sentences as \emph{updates}.

In order to measure the usefulness of a system's updates, we consider
the degree to which the system output reflects the different
aspects of an event.  Events are often composed of a variety of sub-events.  
For example, the Hurricane Sandy
event includes sub-events related to the storm making landfall,
the ensuing flooding, the many transportation issues, amongst many
others.  An ideal system would update the user about each of these
sub-events as they occur, with low latency.  In order to use
language consistent with previous literature, we refer to these
sub-events as the \emph{nuggets} associated with an event.  Nuggets are
defined as a fine-grained atomic sub-event associated with an event.  
We present several example nuggets associated with the Hurricane
Sandy event in Figure \ref{fig:nuggets}.  We describe how these 
nuggets are found in Section \ref{sec:data}.

% Updates should not only be relevant to the event but also salient,
% i.e. worthy of
% reporting.
% For most domains, updates also need to be
% timely and novel, i.e. users need to be informed of changes in the
% event as quickly as possible.
% Finally, in aggregate the updates must comprehensively summarize the
% event.

\begin{figure}
\setlength{\fboxsep}{10pt}
    \fbox{\parbox{6.9cm}{
        -- hurricane force wind warnings are in effect from Rhode Island Sound to Chincoteague Bay

        -- Obama declared an emergency in Maryland and signed an order authorized the Federal Emergency Management Agency to aid in disaster relief

        -- over 5000 commercial airline flights scheduled for October 28 and October 29 were cancelled
 }}
 \caption{Example nuggets from Hurricane Sandy.\label{fig:nuggets}}
\end{figure}


Throughout our treatment of our algorithm, the \emph{salience} 
of an update captures the degree to which it reflects an 
event's nuggets.  Assuming that we have a text representation 
of our nuggets, the salience of an update $u$  with respect to a set of nuggets $N$ is
defined as,
\begin{align}
\operatorname{salience}(u) = \operatorname{max}_{n \in N} 
\operatorname{sim}(u, n) \ref{eq:salience}
\end{align}
where $\operatorname{sim}(\cdot)$ is the semantic similarity such as
the cosine similarity of latent vectors associated with the update and 
nugget text \cite{?}. % We train
% a regression model to predict salience using features derived from
% the updates themselves
% and the document stream.



     % \ref{sec:methods}.  In our version of this problem, we assume that that
     % the system receives one batch of new sentence-segmented documents every
     % hour throughout the period of interest.
\subsection{Update Summarization}

Our system architecture follows a simple pipeline design where each
stage provides an additional level of processing or filtering of the input
sentences.
We begin with an empty update summary $U$.
At each hour we receive a new batch of sentences $b_t$ from the stream
and perform the following actions:
\begin{enumerate}
    \item predict the salience of sentences in $b_t$ (Section~\ref{sec:salpred}),
  \item select a set of exemplar sentences  in $b_t$ by combining clustering with 
      salience predictions (Section~\ref{sec:exsel}),
  \item add the most novel and salient exemplars to $U$ (Section~\ref{sec:upsel}).
\end{enumerate}

The resultant list of updates $U$ is our summary of the event.
% Step 2 uses the affinity propagation algorithm to incorporate the salience
% predictions from step 1. We refer to this summarization
% system as the \textbf{AP+Salience} model.


%\kmcomment{This was as best as I could do on this. These are the example
%    categories from the wikipedia page. However you refer in the language model
%    description to the event type earthquake which is more specific. Also, in
%    the TREC data you refer to event types that are specified in the metadata.
%So I'm wondering how many different categories you have and where they come
%from.}

%\fdcomment{include description of nuggets here.} 
%%\kmcomment{We should jointly
%    discuss what goes in data and what goes in problem definition. I have moved
%    event types here because I think knowing how many types and examples of
%    them would be helpful upfront. I think where nuggets and wikipedia pages go
%is questionable.}

%Figure \ref{alg:temporal-summarization} outlines our general update
%summarization algorithm.  At each hour, the system processes each input 
%sentence batch $S_t$.
%We predict the salience $P$ for all input sentences $s\in S_t$.
%Next, we cluster $S_t$ using the AP clustering algorithm, biased by $P$,
%obtaining a set of exemeplar sentences $E$. Finally we select a subset of $E$
%to be updates and add those to our set of summary updates $U$.
%
%\begin{algorithm}%[H]
% \KwData{ 
%     $S_{\stime}, S_{\stime+1},\ldots,S_{\etime} $ --- time ordered sentence
%     batches\\
% \KwResult{U --- a list of updates, i.e. the update summary} 
%}
% ~\\
% Initialize empty list $U$ of updates\\
%    \For{$t \gets \stime,\ldots,\etime $}{
%        $P \gets \operatorname{PredictSalience}(S_t)$\\
%        $E \gets \operatorname{APCluster}(P, S_t)$ \\
%        $U_t \gets \operatorname{SelectUpdates}(E,P,U)$\\
%        $U \gets U \cup U_t$
%}
% \caption{Temporal Summarization Algorithm}\label{alg:temporal-summarization}
%\end{algorithm}

\subsection{Salience Prediction}
\label{sec:salpred}

\subsubsection{Model}

When evaluating our summarization system
we do not have access to event nuggets. However, we can use the nuggets
in our training data to learn a model to predict salience for sentences
in new events.
For each event in our training data, we sample a set of sentences and  each 
sentence's salience is computed according to Equation \ref{eq:salience}.
This results in training set of set sentences and the target values 
(their salience) to predict.

In order to fit a regression model, the sentences are represented as feature
vectors (described in the next section).
We fit a Gaussian process (GP) regression model to this data.
\cite{rasmussen:gaussian-process-book}.  GP regressors are a
class of data-driven, non-parametric model generalizing the multi-variate
Gaussian to the infinite dimensional setting.  
%GP's are general
%and are state of the art for many regression tasks.  
GP's are non-parametric and rely on a covariance matrix $\kernelMatrix$, measuring the similarity between pairs of instances, in our 
case sentences.  In our experiments, we used a radial basis 
function (RBF) kernel.  
Our features fall naturally into five groups and we use a separate RBF kernel
for each, using the sum of each feature group kernel as the final input
to the GP model.

%include feature table
\begin{figure}[t!]
\begin{tabular}{| l |} 
\hline
\textbf{Basic Features}\\
$\cdot$ Sent. position (normalized by doc. length) \\
$\cdot$ Sent. length \\
$\cdot$ Ratio of punc./caps./lower/upper to \\
$\;\;$ other chars. \\
%$\cdot$ Ratio of caps. to non-caps. chars. \\
%$\cdot$ Ratio of lowercase to other chars. \\
%$\cdot$ Ratio of uppercase to other chars. \\
$\cdot$ \# of caps. words (normalized by \# of words)\\
$\cdot$ \# of Person, Location, Org., etc.\\
$\;\;$ (normalized by \# of words)\\
%$\;\;$ Ordinal, Percent, Money, Set, Misc  N.E. \\
\hline
\textbf{Query Features}\\
$\cdot$ \% of query words coverage/matches\\
$\cdot$ Total event-type synonyms/hypernyms/ \\
$\;\;$ hyponyms coverage/matches.\\
%$\cdot$ Total event-type synonyms/hypernyms/ \\
%$\;\;$ hyponyms matches.\\
\hline
\textbf{Language Model Features}\\
$\cdot$ Avg. token log probability (domain lang. \\
$\;\;$ model)\\
$\cdot$ Avg. token log probability (background \\
$\;\;$ lang. model)\\
\hline
\textbf{Geo-tag Features}\\
$\cdot$ Median document distance to nearest  \\
$\;\;$ location cluster (current hour/prev. hour).\\
%$\cdot$ Median document distance to nearest  \\
%$\;\;$ location clusters (previous hour).\\
$\cdot$ Distance of first location in doc. to nearest \\
$\;\;$ location cluster (current hour/prev. hour).\\
%$\cdot$ Distance of first location in doc. to nearest \\
%$\;\;$ location cluster (previous hour).\\
\hline
\textbf{Temporal Features}\\
$\cdot$ Avg. tf-idf at current time.\\
$\cdot$ Change in avg. tf-idf since previous hour \\
$\;\;$  (up to 24 hours).\\
$\cdot$ Time since query/event start.\\
\hline
\end{tabular}
\caption{Salience model features.}
\end{figure}


\subsubsection{Features}
We want our model to be predictive across different event instances so we avoid lexical features.  Instead, we extract a variety of features including language model scores, geographic relevance, and temporal relevance from each sentence.  

\paragraph{Basic Features}
%KM - Would be good to have quick justification of these features. I added a
%sentence. Feel free to edit or remove.

We employ several basic features that have been used previously in supervised models to rank sentence salience \cite{kupiec1995trainable,conroy2001using}. These include sentence length, the number of capitalized words normalized by sentence length, document position, number of named entities.  
The data stream comprises text extracted from raw html documents;
these features help to downweight sentences that are not actually article 
content (e.g. web page titles, links to other content) or
more heavily weight important sentences (e.g., that appear in
prominent positions such as paragraph initial or article initial).

\paragraph{Query Features}

Query features measure the relationship between the sentence and the event query and type.  These include the number of query words present in the sentence in addition to the number of event type synonyms, hypernyms, and hyponyms as found in WordNet \cite{miller1995wordnet}.  
For example, for event type \emph{earthquake},  we match sentence terms 
``quake'', ``temblor'', ``seism'', and ``aftershock''.

\paragraph{Language Model Features}\label{subsubsec:lm}
Language models allow us to measure the likelihood of a sentence having been 
produced from a particular source.  We consider two types of language model 
features.  The first model is estimated from a corpus of generic news 
articles \ckcomment{(we used the 199?-200? Associated Press and New York Times sections of the Gigaword corpus)}.  
This model is intended to assess the general writing quality (grammaticality, word usage) of an input sentence and helps our model to select sentences
written in the newswire style.  

The second model is estimated from text specific to our event types.  
For each event type we create a corpus of related documents using pages
and subcategories listed under a related Wikipedia category.
For example, the language model for event type `earthquake' is estimated 
from Wikipedia pages under the category \emph{Category:Earthquakes}.  

\fdcomment{stress here that this technique can be applied in situations beyond crisis; any situation where the
expected summary is similar to some previously seen target or some side information.}

These models are intended to detect sentences similar to those appearing in 
summaries of other events in the same category 
(e.g. most earthquake summaries are likely to include higher probability for 
ngrams including the token `magnitude').  

We use the SRILM toolkit to implement a 5-gram Kneser-Ney model for both
the background language and model and the event specific language models.
For each sentence we use the average token log probability under each model
as a feature.


%KM - Note: Someplace the exact list of event types should appear. Probably not
%here.
%KM - I note you have it in a later section but it is labeled as data you use
%to train language models and semantic similarity. I think it would be good to
%have up front in definition of task.


%For both models, we Finally, we extract the percentage of capitalized words,
%and sentence length as features. These last two features also help to
%identify sentences that are less likely to contain relevant content-- overly
%long and heavily capitalized sentences in our corpus were likely to be long
%strings of web-page headlines, section headers, and other irrelevant page
%structure. 

\paragraph{Geographic Relevance Features}
\fdcomment{need introduction here.  what are geographic features?}
There are two challenges to using geographic features. First we do not 
know where the event is and second most sentences do not contain references
to a location.
We address the first issue by extracting all locations from sentences at the
current hour and looking up their latitude and 
longitude using a publicly available geolocation service. 
\fdcomment{is this all documents in the bucket or the filtered set?  is it a biased sample?}
Since the document
stream contains documents that are at least somewhat relevant to the event,
we assume in aggregate the locations should give us a rough area of interest.
The locations are clustered (using affinity propagation and uniform salience)
\fdcomment{affinity propagation undefined}
and we treat the resulting cluster centers
as the event locations for the current time.

To address the second issue, 
we compute geographic relevance features for the document as a whole and all
sentences from that document receive the same feature value.
Using the locations in each document, we compute the median distance to the 
nearest event location. Because document position is a good indicator 
of importance we also compute the distance of the first mentioned
location to the nearest event location. Because some events can move, we also
compute these distances to event locations from the previous hour.



\paragraph{Temporal Relevance Features}
\fdcomment{motivate burstiness; explain what it is before using the expression.}
Our data consists of hourly crawls of online content and so we exploit the temporality of corpus by capturing the burstiness of a sentence, i.e.  the change in word frequency from one hour to the next.``Bursty'' sentences often indicate new and important data. 

We compute the IDF for each hour in our data stream. 
For each sentence, the average TFIDF for the current hour $t$ is taken as a 
feature. Additionally, we use the difference in average TFIDF from time $t$
to $t-i$ for $i = \{1, \ldots, 24\}$ to measure how the TFIDF scores for the 
sentence have changed over the last 24 hours. \fdcomment{unclear.  is it just changing the IDF part of the 
items in the average?}

The final temporal feature is the hours since the event started.
%difference 
%Using these IDF 
%calculations, the avg TFIDF is calculated for
%Let $D_t$ be the set of web pages at time $t$ and let $s = \{w_1,\ldots,w_n\}$ be a sentence from a page $d \in D_t$.  We calculate the 1-hour burstiness of sentence $s$ from document $d$ at hour $t$  as 
%\begin{align*}
%\operatorname{b}_1(s,d,t) = \frac{1}{|s|} \sum_{w \in s} \Bigg( &
%\operatorname{tf-idf}_t(w,d)  \\ & \left. - \frac{\sum_{d^\prime \in D_{t-1}:
%w \in d^\prime } \operatorname{tf-idf}_{t-1}(w,d^\prime)}{|\{d^\prime \in
%D_{t-1}: w \in d^\prime\}|} \right) \end{align*}

%where \begin{align*} \operatorname{tf-idf}_t(w,d) =&
%\log\left(1+\sum_{w^\prime \in d}1\{w=w^\prime\}  \right)\\ & \times
%\log\left(\frac{|D_t|}{1 + \sum_{d^\prime \in D_t}1\{w \in d^\prime\}}\right).
%\end{align*}

%[\textrm{Simpler explanation goes here}]
% 1\{w = w^\prime} %- \operatorname{avg-tf-idf}_{t_{i-1}}(w).
%\end{align*}


%We similarly find the sentence's 5-hour burstiness.  In addition to burstiness, we also include the sentence's average tf-idf and 


% Formally, let $p(f)$ be a distribution over functions where $f$ is any mapping
% of an input space $\mathcal{X}$ to the reals,
%
% $$f: \mathcal{X} \rightarrow \mathcal{R}.$$
% Let the random variable $\mathbf{f} = (f(x_1),\ldots,f(x_n) )$ be
%  an $n$-dimensional vector whose elements are evaluations of the function $f$
% at points $x_i \in \mathcal{X}$.
% We say $p(f)$ is a Gaussian process if for any finite subset
% $\{x_1,\ldots,x_n\} \subset \mathcal{X}$, the marginal distribution over
% that finite subset $p(\mathbf{f})$ has a multivariate Gaussian distribution.
% A GP is parameterized by a mean function $\mu(\mathbf{x})$ and a
% covariance function $K(x,x^\prime)$. Generally, the mean function is simply
% set to 0, leaving the distribution to be completely characterized by the
% kernel function on the data.
%
% In the regression setting, we typically have a response variable $y$ that
% is the sum of our model prediction  and
% some Gaussian noise, i.e. $y = f(x) + \epsilon$ with
% $\epsilon \sim \mathcal{N}(0, \sigma^2)$. When
% $f \sim \operatorname{GP}(\mathbf{0}, \mathbf{K})$, the
% two distributions
% of principal interest are the marginal likelihood
% $p(\mathbf{y}|\mathbf{X}) =
% \mathcal{N}(\mathbf{0},\mathbf{K} + \sigma^2\mathbf{I})$ and the predictive
% distribution,
%
% $$p(\mathbf{y_*}|\mathbf{x_*},\mathbf{X},\mathbf{y}) =
% \mathcal{N}(\boldsymbol{\mu}_*, \boldsymbol{\sigma}^2_*) $$
%
% where $\mathbf{x_*}$ is a new or unseen input, $\mathbf{y_*}$ our predicted
% response, and
% \begin{align*}
% \boldsymbol{\mu}_* & = \mathbf{K_*}(\mathbf{K} + \sigma^2\mathbf{I})^{-1}\mathbf{y} \\
% \boldsymbol{\sigma}^2_* &
% = \mathbf{K}_{**} - \mathbf{K}_*(\mathbf{K} + \sigma^2\mathbf{I})^{-1}
% \mathbf{K}_*^T + \sigma^2\\
% \end{align*}.
%
% Here $\mathbf{K}_* = K(\mathbf{x}_*, \mathbf{X})$, and
% $\mathbf{K}_{**} = K(\mathbf{x}_*, \mathbf{x}_*)$.
%
%

\subsection{Exemplar Selection}
\label{sec:exsel}

We combine the output of our salience prediction model with the affinity
propagation algorithm to identify a set of exemplar sentences 
for each input batch. 
Affinity propagation (AP) is a clustering algorithm
that identifies a subset of datapoints as exemplars and forms clusters
by assigning the remaining points to one of the exemplars. AP attempts to 
maximize the net similarity objective 
\[ \mathcal{S} = \sum_{i : i \neq e_i}^n \operatorname{sim}(i,e_i) 
+ \sum_{i : i = e_i}^n \operatorname{pref}(e_i)  \]
where $e_i$ is the exemplar of the $i$-th data point, and functions
$\operatorname{sim}$ and $\operatorname{salience}$ express the pairwise 
similarity
of data points and the apriori preference of a data point to be an exemplar
respectively. 
AP differs from other $k$-centers algorithms in that it simultaneously 
considers all data points as exemplars, making it less prone to finding
local optima as a result of poor initialization. Furthermore, the 
second term in $\mathcal{S}$ incorporates the individual importance of 
data points as candidate exemplars; most other clustering algorithms only make
use of the first term, i.e. the pairwise similarities between data points.
 
%There are very 
%few restrictions on the nature of the similarities and preferences
%other than that they be real-valued.

AP has several useful properties and interpretations. Chiefly, the number
of clusters $k$ is not a model hyper-parameter. Given that our task requires
clustering many batches of streaming data, searching for an optimal $k$ 
would be computationally prohibitive. With AP, $k$ is determined by the
similarities and preferences of the data. Generally lower preferences will
result in fewer clusters.  


%In our summarization system we use the output of our salience model as 
%the preferences, i.e.
%$\operatorname{salience}(s) = \operatorname{pref}(s)$; for the similarity 
%function we use semantic similarity. 
Recall that $\operatorname{salience}(s)$
is a prediction of the semantic similarity of $s$ to information about the 
event be summarized, i.e. the set of event nuggets.
Intuitively, when maximizing objective function $\mathcal{S}$, AP must balance
between best representing the input data and representing the most salient
input. Additionally, when the level of input is high but the salience
predictions are low, the preference term will guide AP toward a solution
with fewer clusters; vice-versa when input is very salient on average but
the volume of input is low. The adaptive nature of our model differentiates
our method from most out update summarization systems.




\subsection{Update Selection}
\label{sec:upsel}

\textbf{Salience Filter } To 
ensure that only the most salient updates are selected we apply a minimum
salience threshold;
after exemplar sentences have been identified, any exemplars whose salience is 
less than $\lambda_{sal}$ are removed from consideration. 

\textbf{Novelty Filter } Next,
to prevent adding updates that are redundant, we filter out exemplars
that are too similar to previous updates.
The exemplars are examined
sequencially in order of decreasing salience and  a similarity threshold 
is applied, where the exemplar is ignored if its
maximum semantic similarity to any previous updates in the summary is
greater than $\lambda_{sim}$.

Exemplars that pass these thresholds are added selected as updates and added
to the summary.







\section{Run Submissions}

We submitted three different runs for official evaluation. The first system,
(AP+) used AP clustering without salience predictions or redundancy penalty.
The clustering was run every hour, and all non-singleton exemplars were 
taken as the updates.

Our second submission (AP+Sal+) uses our salience prediction model, with the
AP clustering to select sentences. We do not penalize for redundancy.

Our final run (AP+Sal+Red-) is the same as the previous run but the 
redundancy penalty is applied.

\section{Results}

Overall it appears that (AP+Sal+) is our best performing model of the three
submissions. One of the notable differences is in the average number of 
updates per event. (AP+Sal+) was by far the most terse system, with 
383.2 updates/event compared to 5997.8 and 1070.7 for (AP+) and (AP+Sal+Red-)
respectively.
Despite not returning many updates, $\mathbb{E}$[Latency Gain] was high enough
on average to outperform on the F1 score compared to the other two systems.
The differences in F1 from the baseline run for both (AP+Sal+) and 
(AP+Sal+Red-) are statistically significant; 
this is empirical validation of our salience modeling efforts.

\begin{figure}
\centering
\begin{tabular}{| c | c | c | c |}
\hline
\textbf{Run} & \textbf{nE[Lat. Gain]} & \textbf{Lat. Comp} & \textbf{F1} \\
\hline
AP+ & 0.0222  & 0.5777 & 0.0403\\
\hline
AP+Sal+ & 0.0751 &  0.4139 & $\mathbf{0.1162}$\\
\hline
AP+Sal+Red- & 0.0375 & 0.3413 & 0.0602\\
\hline
\end{tabular}
\caption{Submission results, averaged over 2014 TREC TS events.}
\end{figure}

Unfortunately, there is still much work to be done with our redundancy 
component. (AP+Sal+Red-) returned an order of magnitude more updates on 
average than (AP+Sal+) but still achieved comparatively lower comprehensiveness
scores. The redundancy penalty is perhaps too aggressive, driving the 
clustering to focus on very novel input spaces---so novel as to be 
irrelevant. 

A related issue with our current model is that our salience predictions
are relative from the point of view of the clustering algorithm. If all inputs
have low salience predictions, the AP clustering will still find exemplars, 
albeit from the most salient inputs. This becomes especially problematic 
with the redundancy component, as with every timestep, we are evaluating 
riskier and riskier sets of more novel sentences, and being forced to return
something. This behavior can be observed especially in the updates from the
Chelyabinsk meteor event, where our document retrieval stage was 
under-performing and the predicted salience of all updates was quite low.
This would suggest implementing some minimum threshold of salience. We
initially experimented with such a threshold but found tuning it across 
events of different magnitudes to be difficult. 


\bibliographystyle{abbrv}
%\bibliography{sigproc}  % sigproc.bib is the name of the Bibliography in this
%case
\bibliography{cites.bib}
% You must have a proper ".bib" file and remember to run: latex bibtex latex
% latex to resolve all references
%
% ACM needs 'a single self-contained file'!
%
%APPENDICES are optional \balancecolumns \balancecolumns % GM June 2007 That's
%all folks!
\end{document}
